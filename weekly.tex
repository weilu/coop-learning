\documentclass[a4paper]{article}
\usepackage{natbib}
\usepackage{url}

\title{Weekly Report}

\date{May 3, 2019}
\author{Wei Lu}

\begin{document}
\maketitle

\section*{Week: July 28th - Aug 1st}

\subsection*{Goals}

\begin{enumerate}
  \item Find out why PAC without sampling and original top covering are producing different results
  \item Implement brute force search and split algorithm to discover stable parition
  \item More writing
\end{enumerate}

\subsection*{Activities}

\begin{enumerate}
  \item Two algorithm outputs are different because PAC implementation is buggy - it shouldn't consider coalitions containing players that have already been removed. Fixed.
  \item Implemented a search and split algorithm to find stable partition
  \item Fixed core stability check - the ``better-off'' check was too weak, it completely ignored unknown coalition for a player, it should have checked through the complete preference list instead
  \item Make top responsive check more efficient by only checking known coalitions
  \item Discovered that the derived Knesset game isn't top responsive
  \item Report: 1 page
\end{enumerate}

\subsection*{Meeting Agenda: Aug 2nd}

\begin{enumerate}
  \item How to make Knesset game top responsive?
  \item Does top covering algorithm work on non-top responsive games too?
  \item Discuss search split algo results
  \item Is it meaningful to find solution that maximizes social welfare in our case?
\end{enumerate}

\section*{Week: July 17th - July 28th}

\subsection*{Goals}

\begin{enumerate}
  \item Get the latest data: redownload data in case data has been updated since last download
  \item Evaluate data quality: find out which endpoint provide more consistent data within itself
  \item Literature review: Coalition structure generation: A survey https://www.sciencedirect.com/science/article/abs/pii/S0004370215001198
  \item Run 100 times on 3/4 and participations accounted see if the generated structure is stable
  \item Start writing
\end{enumerate}

\subsection*{Activities}

\begin{enumerate}
  \item Redownloaded data - the votes data indeed have been updated since last download
  \item Inconclusive which endpoint has higher quality data as they both are consistent within
  \item Read paper: Coalition structure generation: A survey
  \item Executed PAC algo with sampling of 3/4 of bills - the output structures are stable, but full of singletons when restriction loop is activated (not sure why)
  \item Wrote one paragraph about our data set
\end{enumerate}

\subsection*{Meeting Agenda: July 29th}

\begin{enumerate}
  \item What to do with data inconsistency?
  \item What could have caused the singletons when sampling with restrictive loop enabled?
  \item Clarify optimized dynamic programming algorithm in the survey paper
  \item Is it meaningful to find solution that maximizes social welfare in our case?
\end{enumerate}

\section*{Week: July 12th - July 16th}

\subsection*{Goals}

\begin{enumerate}
  \item Verify data pre-processing
  \item Tweak value function to count abstained as paritcipation and observe result change
  \item literature review: Coalition structure generation: A survey https://www.sciencedirect.com/science/article/abs/pii/S0004370215001198
  \item run 100 times on 3/4 and participations accounted see if the generated structure is stable
\end{enumerate}

\subsection*{Activities}

\begin{enumerate}
  \item Went through data download and preprocessing code; fixed bug that produced excessive number of 0s in votes.csv
  \item Download additional data from bill summary endpoint to check sanity of member-bill vote data - inconsistency discovered
  \item Count ``abstained'' as participation - the results are mostly the same
\end{enumerate}

\subsection*{Meeting Agenda: July 17th}

\begin{enumerate}
  \item How to resolve data inconsistency?
  \item More agorithms to implement?
\end{enumerate}

\section*{2 Weeks: June 26th - July 11th}

\subsection*{Goals}

\begin{enumerate}
  \item Coding and experimentation: increase size of average coalitipn for Knesset output
  \begin{itemize}
    \item Fix value function to calculate majority only once per bill
    \item Tweak sample size and value function
    \begin{enumerate}
      \item change value function to account for participation
      \item change value function to account for how close the victory was
      \item give losing coalition some small value
    \end{enumerate}
  \end{itemize}
\end{enumerate}

\subsection*{Activities}

\begin{enumerate}
  \item Pre-calculated valuations and coalition for every player and every bill
  \item Programmatically verified core stability of Knesset output
  \item Tried smaller sample size and the output average coalition size is still small - 1
\end{enumerate}

\subsection*{Meeting Agenda: July 12th}

\begin{enumerate}
  \item How to use additional data?
  \item More tweaks?
  \item How to measure output partition's closeness to party affiliation?
\end{enumerate}

\section*{Week: June 20th - 25th}

\subsection*{Goals}

\begin{enumerate}
  \item Coding:
  \begin{itemize}
    \item Implement PAC top covering algorithm (algorithm 1)
  \end{itemize}
\end{enumerate}

\subsection*{Activities}

\begin{enumerate}
  \item Completed first implementation of PAC top covering algorithm
  \item Executed the algorithm on knesset data
\end{enumerate}

\subsection*{Meeting Agenda: June 26th}

\begin{enumerate}
  \item Clarification:
  \begin{itemize}
    \item All preferences ranked after self is solely depends on size? Yes
    \item How to treat missing preferences? They are worse than known options
    \item If total for votes is more than against, but less than half the parliment, does the bill pass? Yes
    \item Do we need to recalculate majority each time any player is removed? No
  \end{itemize}
  \item Most output coalitions are of size 1
  \item Construct more test cases
  \item How to deal with uncertainty in testing?
\end{enumerate}

\section*{Week: June 10th - 20th}

\subsection*{Goals}

\begin{enumerate}
  \item Literature review: understand PAC top covering algorithm
  \item Coding:
  \begin{itemize}
    \item verify top covering algorithm efficiency
    \item Check top covering algorithm weakly better off edge case
  \end{itemize}
\end{enumerate}

\subsection*{Activities}

\begin{enumerate}
  \item Added additional manual test cases provided by Alan
  \item Evaluated program efficiency: game with 142 players generated in: 14s; top covering algo completed in: 18s.
\end{enumerate}

\subsection*{Meeting Agenda: June 21th}

\begin{enumerate}
  \item Clarify PAC top covering algorithm
\end{enumerate}

\section*{2 Weeks: May 9th - May 22nd}

\subsection*{Goals}

\begin{enumerate}
  \item Literature review: more on hedonic games
  \begin{itemize}
    \item Handbook of Computational Social Choice, Chapter 15 Hedonic Games \cite{aziz_savani_moulin_2016}
    \item Researching with whom? stability and manipulation \cite{ALCALDE2004869}
  \end{itemize}

  \item Coding: implement top covering algorithm

\end{enumerate}

\subsection*{Activities}

\begin{enumerate}
  \item Read Handbook of Computational Social Choice, Chapter 15 Hedonic Games \cite{aziz_savani_moulin_2016}
  \item Implemented top covering algorithm with handcrafted test cases
  \item Review \& referenced relevant part of \cite{ALCALDE2004869}
\end{enumerate}

\subsection*{Meeting Agenda: June 10th}

\begin{enumerate}
  \item Review top covering algorithm implementation
\end{enumerate}

\section*{Week: May 3rd - May 8th}

\subsection*{Goals}

\begin{enumerate}
  \item Project setup
  \begin{itemize}
    \item overleaf for doc collaboration: https://www.overleaf.com/project/5ccbf7fab95406669496706a
    \item github for code versioning and review: https://github.com/weilu/coop-learning
  \end{itemize}

  \item Literature review: understanding the key concepts and algorithms
  \begin{itemize}
    \item PAC learning
    \item Top Responsive games
    \item Learning Hedonic Games \cite{ijcai2017-380} algorithm 1
  \end{itemize}

\end{enumerate}

\subsection*{Activities}

\begin{enumerate}
  \item Understood PAC learning basics \cite{Valiant:1984:TL:1968.1972}
  \item Shattering and VC dimension \cite{vc}
\end{enumerate}

\subsection*{Meeting Agenda: May 9th}

\begin{enumerate}
  \item Walk through algorithm 1, why it works
  \item Clarify questions from Learning Hedonic Games \cite{ijcai2017-380}
\end{enumerate}

\bibliographystyle{ACM-Reference-Format}  % do not change this line!
\bibliography{weekly-references}  % put name of your .bib file here


\end{document}

