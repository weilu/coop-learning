\chapter{Conclusion}
\label{ch:concl}

To review, in this thesis we set out to cluster Israeli parliament members
based on incomplete preferences derived from their votes on legislative bills.
We take a data-driven approach while account for strategic behavior of
parliament members as players of hedonic coalition formation games.
In comparison, the clustering and community detection methods commonly used in
the machine learning/AI community are also data-driven but fail to take into
consideration individual player's utility and strategic behaviors.
Whereas the game theory and computational social choice community has primarily
studied the theoretical aspect of hedonic coalition formation games;
the limited number of studies on the application aspect of hedonic games are
``prescriptive'' and often with simulated data.
Our study bridges the gap by bringing real-world data of scale and with ground
truth, the Knesset voting data on over 7000 legislative bills, together with
PAC stable solutions for hedonic games.

We demonstrate that the top three performing PAC models, according to the
information theoretical measures against party affiliations, not only are able
to recover the overall structure of government vs.\ opposition groups, but also
are more coherent compared to machine learning models such as $k$-means and
Stochastic Block Models.
By comparing the PAC setting and full-information setting for both the friends
and Boolean models, we observe that through sampling, PAC models produce
partition results that are more robust --- sampling help ``dampen'' the
model's sensitivity to the definition of friends in the case of friends model,
and ``smooth'' the effect of individual bills with significant cross-party
support/disapproval in the case of Boolean model.

One limitation of our study is that it is a case study based off a single
dataset.
A natural next step is to apply the three top performing models, namely
selective friends, selective enemies, and Boolean models on other parliaments,
such as that of the Netherlands which also has a coalition government;
the United States congress could also be interesting as one could test if the
model reveals any regional effect.
Another research direction is to apply other hedonic uncertainty models, such
as the reinforcement learning model by \citenameyear{Chalkiadakis2004} for
finding a Bayesian core, on the Knesset dataset, and compare the resulting
partitions with that of our PAC models using the same criteria.
