\chapter{Conclusion}
\label{ch:concl}

In this thesis we set out to cluster Israeli parliament members
based on incomplete preferences derived from their votes on legislative bills.
We take a data-driven approach while accounting for strategic behavior of
parliament members as players of hedonic coalition formation games.
By comparison, the clustering and community detection methods commonly used in
the machine learning/AI community fail to consider players' preferences and strategic behavior.
Whereas the game theory and computational social choice community has primarily
studied the theoretical aspect of hedonic coalition formation games;
the limited number of studies on the application of hedonic games are
``prescriptive'' and often use only simulated data.
Our study bridges the gap through conducting a ``descriptive'' study of hedonic
game theoretical models using real-world data of scale and with ground truth ---
the Knesset voting data on over 7000 legislative bills with party affiliations as
the baseline partition.

We demonstrate that the top three performing PAC models, according to the
information theoretical measures against party affiliations, are not only able
to recover the overall structure of government vs. opposition groups, but also
are more coherent compared to machine learning models such as $k$-means and
Stochastic Block Models.
By comparing the PAC and full-information settings for both the friends
and Boolean models, we observe that through sampling, PAC models produce
partition results that are more robust --- sampling helps ``dampen'' the
model's sensitivity to the definition of friends in the case of friends model,
and ``smooth'' the effect of individual bills with significant cross-party
support/disapproval in the case of Boolean model.

One limitation of our study is that it is a case study based off a single
dataset.
A natural next step is to apply the three top performing models, namely
selective friends, selective enemies, and Boolean models on other parliaments,
such as that of the Netherlands which also has a coalition government.
The Brazilian parliament data is public accessible\footnote{The data API is at
\url{https://dadosabertos.camara.leg.br/swagger/api.html}}, and is studied by
\citenameyear{Peixoto} using the stochastic block model.
The United States congress could also be interesting as one could test if the
model reveals any regional effect.
Another research direction is to apply other hedonic uncertainty models, such
as the reinforcement learning model by \citenameyear{Chalkiadakis2004} for
finding a Bayesian core, on the Knesset dataset, and compare the resulting
partitions with that of our PAC models using the same criteria.