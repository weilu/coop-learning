\SetPicSubDir{ap-Boolean}
\SetExpSubDir{ap-Boolean}

\chapter{Alternative Formulation of Boolean Hedonic Games}
\label{append:alternative_boolean}
\vspace*{2em}

We can alternatively derive preferences from parliament members' past voting
activities using the following idea:
whenever a member votes with either ``for'' (or ``against''), they not only
express their preference to be with other members whose votes are the same
as theirs, but also not to be with the group that voted differently.

Recall Example~\ref{example:votes_boolean}:
From bill 1, we have $\{1, 3\} \succ_1 \{1, 2\}$, $\{2\} \succ_2 \{1, 2, 3\}$,
$\{1, 3\} \succ_3 \{2, 3\}$.
From bill 2, we have $\{2, 3\} \succ_2 \{2\}$, $\{2, 3\} \succ_3 \{3\}$.
Player 1 is not involved as they voted abstained.
Since there is nobody voted ``2 - against'', both player 2 and player 3 dislike
joining an empty coalition.
From bill 3, we have $\{1\} \succ_1 \{1, 2, 3\}$, $\{2, 3\} \succ_2 \{1, 2\}$,
$\{2, 3\} \succ_3 \{1, 3\}$.

In order to combine the preferences from bill 1, 2 and 3, we need to handle
the cases when a coalition is both liked and disliked by a player, for example,
player 3 likes $\{1, 3\}$ according to bill 1 but dislikes it according to bill 3;
player 3 also likes $\{2, 3\}$ according to bill 2 and 3 but dislikes it
according to bill 1.
We tally the number of times a given coalition is liked and disliked by a
player and rule according to the preference with more occurrences.
When there are equal number of likes and dislikes, we break tie in favor of dislike.
As such, in the above Example~\ref{example:votes_boolean}, player 3 dislikes
$\{1, 3\}$ and likes $\{2, 3\}$.

Therefore we can merge the preferences derived from bill 1, 2, and 3:

player 1: $\{1, 3\} \sim \{1\} \succ_1 \{1, 2\} \sim \{1, 2, 3\}$

player 2: $\{2, 3\} \succ_2 \{2\} \sim \{1, 2\} \sim \{1, 2, 3\}$

player 3: $\{2, 3\} \succ_3 \{1, 3\} \sim \{3\}$

The merged preferences is not guaranteed to be a complete preference profile;
as shown in the merged preferences derived from Example~\ref{example:votes_boolean},
player 3's preference over the grand coalition is unknown.

We experimented with treating missing preferences both as satisfactory and
then unsatisfactory.
The resulting partitions share one large common coalition which coincides with
the bill with multi-party support.
The default-unsatisfactory preference produces many singleton coalitions;
in comparison, the default-satisfactory preference lumps players together as
many as possible.

\begin{figure}[h!]
  \centering
  \begin{minipage}{0.48\linewidth}
    \centering
    \includegraphics[width=\linewidth]{\Pic{png}{alt_boolean_default_dislike}}
    \caption{Alternative Boolean default-unsatisfactory model}
    \label{alternative_boolean:fig:alt_boolean_default_dislike}
  \end{minipage}\hfill
  \hspace{.03\linewidth}
  \begin{minipage}{0.48\linewidth}
    \centering
    \includegraphics[width=\linewidth]{\Pic{png}{alt_boolean_default_like}}
    \caption{Alternative Boolean default-satisfactory model}
    \label{alternative_boolean:fig:alt_boolean_default_like}
  \end{minipage}
\end{figure}

Either way, both partitions from this formulation have high variation of information.

% TODO: add min/max/best for comparison
\begin{table}[h!]
\centering
\begin{tabular}{|c|c|c|}
\hline
       & default-unsatisfactory & default-satisfactory \\ \hline
Variation of Information & 3.487 & 3.716 \\
Adjusted Mutual Information & 0.018 & 0.005 \\
\hline
\end{tabular}
\end{table}
