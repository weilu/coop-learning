\SetPicSubDir{ch-Preliminaries}
\SetExpSubDir{ch-Preliminaries}

\chapter{Preliminaries}
\label{ch:preliminaries}
\vspace{2em}

In this chapter, we recall some definitions relating to hedonic games,
coalition structures, stability solution concepts, and PAC stability.
These technical details are important foundation for our data driven experiments.

\section{Hedonic Games}
\label{sec:hedonic_game}


A \textit{hedonic game} is a pair $(N, P)$, where $N$ is a finite set of players
$\{1, \cdots, n\}$ and $P$ is a \textit{preference profile} consisting of
preference relations $\succeq_i$ for every player $i \in N$:
$P = (\succeq_1, \cdots, \succeq_n)$.
A \textit{preference relation} $\succeq_i$ is a reflective, complete, and
transitive binary relation on $\mathcal{N}_i$, where $\mathcal{N}_i$ is the set
of all non-empty subset of $N$ that includes player $i$;
i.e., $\mathcal{N}_i = \{S \subseteq N: i \in S, S \neq \emptyset \}$

Consider a parliament; who are ``natural allies''? One way to model this is by
assuming that each member has a preference relation over various groups they
wish to be part of; 
our objective is to identify partitions
(also known as \textit{coalition structures}) that satisfy certain desirable
properties, such as stability.

We are learning preferences from data, but still need to make certain assumptions about player utility models.
We now describe the two classes of preference models our study is built upon.

\subsection{Top Responsive Preferences}
\label{subsec:top_responsive_preferences}
The intuition behind top responsiveness is that every player derives their
utility from a most preferred subset of players from the coalition they belong to.
If two coalitions, with one containing the other, yield the same utility for
a player, the tie is broken in favor of the smaller coalition.

In the context of our dataset, it captures the following idea:
politicians care about whose votes they stand with;
they want to vote with other politicians they like.
If they can manage to pass a bill with fewer members involved,
it is more efficient therefore more preferable.

A player $i$'s most preferred sets of coalitions are called \textit{choice sets}:
$\Ch(i, S) = \{S' \subseteq S: (i \in S') \wedge (S' \succeq_i S'' \forall S'' \subseteq S)\}$.
The unique choice set in $\Ch(i, S)$ is denoted as $\ch(i, S)$.
A {\em top responsive} preference profile requires that for any player $i \in N$,
and any coalition that may contain player $i$: $S, T \in \mathcal{N}_i$:
\begin{enumerate}
  \item $|\Ch(i, S)| = 1$.
  \item if $\ch(i, S) \succ_i \ch(i, T)$ then $S \succ_i T$
  \item if $\ch(i, S) = \ch(i, T)$ and $S \subset T$ then $S \succ_i T$
\end{enumerate}

\begin{example}
\label{example:not_top_responsive_pref}
  Consider the classic ``love triangle'' preference profile, where player 1 wants
  to be with player 2, player 2 wants to be with player 3, and player 3 wants to
  be with player 1. All players prefer two-player coalitions over the three-player
  coalition over being alone.
  The resulting preference profile is a hednoic game however not a top responsive
  game:

  player 1: $\{1, 2\} \succ_1 \{1, 3\} \succ_1 \{1, 2, 3\} \succ_1  \{1\}$

  player 2: $\{2, 3\} \succ_2 \{1, 2\} \succ_2 \{1, 2, 3\} \succ_2  \{2\}$

  player 3: $\{1, 3\} \succ_3 \{2, 3\} \succ_3 \{1, 2, 3\} \succ_3  \{3\}$

  It does not satisfy top responsiveness because the grand coalition $\{1, 2, 3\}$
  contains every player's top choice, which is their choice set when all players
  are considered, but is outranked by another two-player coalition every time.
\end{example}

\begin{example}
\label{example:top_responsive_pref}
  If we modify Example~\ref{example:not_top_responsive_pref} by making every player
  rank the three-player coalition as their second most preferred coalition, the
  resulting preference profile is top responsive:

  player 1: $\{1, 2\} \succ_1 \{1, 2, 3\} \succ_1 \{1, 3\} \succ_1  \{1\}$

  player 2: $\{2, 3\} \succ_2 \{1, 2, 3\} \succ_2 \{1, 2\} \succ_2  \{2\}$

  player 3: $\{1, 3\} \succ_3 \{1, 2, 3\} \succ_3 \{2, 3\} \succ_3  \{3\}$

\end{example}

The description of a generic top responsive preference profile is exponentially
large in the number of players, as each player needs to rank all possible
coalitions they may belong to.
In our implementation and experiments in Chapter~\ref{ch:hedonic}, we will
explore top responsive models in both its generic form
(Subsection~\ref{subsec:handcrafted_value_function}) and a specific sub-class of
this preference model named Appreciation of Friends
(Subsection~\ref{subsec:appreciation_of_friends}).

\subsection{Bottom Responsive Preferences}
\label{subsec:bottom_responsive_preferences}

Bottom responsive games can be viewed as the counterpart of top responsive games,
where a player's utility is derived from the absence of disliked players.
In the context of parliment voting data, it models the idea that politicians wants
to avoid voting with those whose ideologies they disagree with.

Similiar to top responsive game's choice sets, bottom responsive game formalizes
``most disliked players'' with the concept of \textit{avoid sets};
$\Av(i, S)$ denotes the avoid sets of player $i$ in coalition $S$: $\Av(i, S) =
\{S' \subseteq S: (i \in S') \wedge (S' \preceq_i S'' \forall S'' \subseteq S)\}$

For a game to satisfy bottom responsiveness, two conditions are required:

\begin{enumerate}
  \item if for all $S' \in \Av(i, S)$ $T' \in \Av(i, T)$, $ S' \succ_i T'$
    then $S \succ_i T$
  \item if $\Av(i, S) \cap \Av(i, T) \neq \emptyset$ and $|S| \geq |T|$
    then $S \succeq_i T$
\end{enumerate}

Different from top responsive game which breaks ``ties'' in favor of
smaller coalitions, condition 2 of bottom responsive game breaks ``ties'' in
favor of larger coalitions.

If in addition to the above two conditions, a game also satisfies
$|\Av(i, S)| = 1$ for all $i \in N$ and all $S \in \mathcal{N}_i$,
then it is strongly bottom responsive. The unique avoid set in $\Av(i, S)$
is denoted $\av(i, S)$.

\begin{example}
\label{example:bottom_responsive_pref}
  If we invert the ``love triangle'' preference profile to ``despise triangle'',
  where player 1 wants to avoid being with player 3,
  player 2 wants to avoid player 1, and player 3 wants to avoid player 2.
  We can formulate the preference profile as follow to be stronly bottom
  responsive:

  player 1: $\{1, 2\} \succ_1 \{1\} \succ_1 \{1, 2, 3\} \succ_1 \{1, 3\}$

  player 2: $\{2, 3\} \succ_2 \{2\} \succ_2 \{1, 2, 3\} \succ_2 \{1, 2\}$

  player 3: $\{1, 3\} \succ_3 \{3\} \succ_3 \{1, 2, 3\} \succ_3 \{2, 3\}$
\end{example}

We will discuss applying the bottom responsive preference model on parliment
voting data in Chapter~\ref{ch:hedonic}, Section~\ref{sec:aversion_to_enemies}.

\subsection{Boolean Preferences}
\label{subsec:boolean_preferences}
Boolean preference model assumes dichotomous preferences of every player ---
a player view any coalition she may belong to as either satisfactory or
unsatisfactory.
A player is indifferent among all satisfactory coalitions and indifferent
among all unsatisfactory coalitions, however strictly prefers any satisfactory
coalition over any unsatisfactory coalition.

\begin{example}
\label{example:boolean_pref}
  The following preference profile describes a Boolean hedonic game where
  player 1 is only happy with the grand coalition, player 2 only likes coalitions
  of size 2, while player 3 only finds it satisfactory when she is alone:

  player 1: $\{1, 2, 3\} \succ_1 \{1, 2\}, \{1, 3\}, \{1\}$

  player 2: $\{1, 2\}, \{2, 3\} \succ_2 \{1, 2, 3\}, \{2\}$

  player 3: $\{3\} \succ_3 \{1, 2, 3\}, \{1, 3\}, \{2, 3\}$
\end{example}

A Boolean preference profile is not necessarily top or bottom responsive --- 
Example~\ref{example:boolean_pref} is neither.
Player 3 has choice sets
$\ch(3, \{1, 2, 3\}) = \ch(3, \{1, 3\}) = \ch(3, \{2, 3\}) = \{3\}$ and avoid sets 
$\av(3, \{1, 2, 3\}) = \{1, 2\}, \av(3, \{1, 3\}) = \{1\},
\av(3, \{2, 3\}) = \{2\}$.
To satisfy top responsiveness, player 3 would prefer $\{1, 3\}, \{2, 3\}$ over
$\{1, 2, 3\}$; to satisfy bottom responsiveness, player 3 would prefer
$\{1, 2, 3\}$ over $\{1, 3\}, \{2, 3\}$, however in Example~\ref{example:boolean_pref} she is indifferent among these three coalitions.

When we observe voting data of parliament members, Boolean preference model
allows us to simply group those members whose votes are the same into a coalition
for any given bill, and label this coalition as satisfactory for everyone who is
a part of.
Intuitively it reflects the idea that a politician is approval of any group they
have ever voted with.
We will investigate Boolean preference model with our empirical voting data in
Chapter~\ref{ch:hedonic} Section~\ref{sec:boolean_hedonic_game}.


\section{Stability Concepts}
\label{sec:stability_concepts}
We are particularly interested in models that produce partitions that satisfy
strategic considerations; specifically, we focus our attention on stability
concepts that capture the idea that no group of players can be better off by
leaving and forming their own coalition.
Group based stability notions are stronger than those based on individual deviation,
such as Nash stability, because individual based stability concepts do not account
for the disutility of the group an individual player is joining or leaving.

\subsection{Core \& Strict Core}
\label{subsec:core_strict_core}
A coalition $S \subseteq N$ \textit{strongly blocks} a coalition structure $\pi$
if every player $i \in S$ strictly prefers $S$ over its current coalition $\pi(i)$;
a coalition structure $\pi$ is \textit{core stable} when there is no strongly
blocking coalition $S$.
A coalition $S \subseteq N$ \textit{weakly blocks} a coalition structure $\pi$
if every player $i \in S$ weakly prefers $S$ over its current coalition $\pi(i)$
and there exists at least one player $j \in S$ who strictly prefers $S$
over $\pi(j)$; a coalition structure is strictly core stable when there is no
weakly blocking coalition.

Example~\ref{example:not_top_responsive_pref} does not have any core stable
partition: the grand coalition $\{1, 2, 3\}$ is strongly blocked by any two-player
coalition; the partition made of a two-player coalition and a one-player
coalition is strongly blocked by another two-player coalition, for example,
$\pi = \{\{1, 2\}, \{3\}\}$ is strongly blocked by $\{2, 3\}$; the partition made
of single players is strongly blocked by any two-player coalition.

Example~\ref{example:top_responsive_pref}, on the other hand, has a strict core
stable partition: $\{\{1, 2, 3\}\}$ as no subset of players can be better off by
deviating together.
Similarly Example~\ref{example:bottom_responsive_pref} also has a strict core
stable partition which is $\{\{1\}, \{2\}, \{3\}\}$.

\subsection{Strong Nash \& Strict Strong Nash}
\label{subsec:strong_nash_strict_strong_nash}
Next we cover two solution concepts with even stronger notions of stability
based on group deviation.
If a partition $\pi' \neq \pi$ exists with movement of players $S \subseteq N$
and $S \neq \emptyset$ (denoted as $\pi \xrightarrow{S} \pi'$), where
$\forall i \in S$, $\pi'(i) \succ_i \pi(i)$,
and $\forall j \in N\text{\textbackslash}S$, $\pi'(j) = \pi(j)$,
then $S$ strongly Nash blocks $\pi$.
A partition that admits no strongly Nash blocking set $S \subseteq N$ is said
to be strong Nash stable (SNS).
A non-empty set of players $S \subseteq N$ weakly Nash blocks $\pi$ if
$\forall i \in S$, $\pi'(i) \succeq_i \pi(i)$ and $\exists j \in S$,
$\pi'(j) \succ_j \pi(j)$.
A partition that admits no weakly Nash blocking set $S \subseteq N$ is said to
be strict strong Nash stable (SSNS).

Strict core stable solutions for Example~\ref{example:top_responsive_pref}
and Example~\ref{example:bottom_responsive_pref} are also SSNS.

\paragraph{}
A top responsive preference profile guarantees the existence of a strict core partition.
Moreover, top responsiveness also warrants a strict strong Nash stable partition
if all players preference relations are \textit{mutual} ––– for all $i, j \in N$,
for all coalition that contains $i$ and $j$:
$S \in \mathcal{N}_i \cap \mathcal{N}_j$, $i \in ch(j, S)$ if and only if
$j \in ch(i, S)$.
This means if we can model empirical data with mutual preferences,
we can achieve the strongest notion of group based notion of stability.

Both bottom responsive and boolean preference profiles warrant the existence of
a core stable coalition structure.
It is such stability guarantees that motivate us to choose these specific set
of hedonic models for our experiments.
Varying degrees of stability notions also allow us to compare how they affect
the qualities of the resulting partitions on real-world voting data.


\section{PAC Learning \& PAC Stability}
\label{sec:pac_learning_pac_stability}
Probably Approximately Correct (PAC) learning is the canonical framework for
provable probabilistic approximations to functions.
In this framework, a learner receives samples and selects a function from a class
of possible functions.
The selected function is called the \textit{hypothesis}, which should be likely
to predict new samples from the same distribution.
A good probabilistic approximation means that with probability of at least
$1 - \delta$, the selected function's output has an average error less than or
equal to $\varepsilon$, where $0 < \varepsilon, \delta < 1$.
A hypothesis class is efficiently PAC learnable if such a good probabilistic
approximation can be produced by some algorithm that has both running time and
input sample size be polynomial in $n$, $\frac{1}{\varepsilon}$, and
$\log{\frac{1}{\delta}}$.

\citenameyear{ijcai2017-380} define that a class of hedonic games is PAC stablizable
if there exists an algorithm that produces a partition $\pi$ such that
$\Pr_{S\sim D}[\text{S core blocks } \pi] < \varepsilon$;
both the number of samples required by the algorithm to provide this PAC guarantee,
and the running time of producing a consistent solution are polynomial in $n$,
$\frac{1}{\varepsilon}$, and $\log{\frac{1}{\delta}}$.
In the same paper, \citenameyear{ijcai2017-380} also show that top responsive games
are efficiently PAC stablizable even though they are not PAC learnable.
A PAC stable partition can be computed with Algorithm~\ref{alg:pac_top_covering}
which we will describe in detail in Chapter~\ref{ch:hedonic}
Section~\ref{sec:top_responsive_game} Subsection~\ref{subsec:algorithms}.
