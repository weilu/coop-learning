\SetPicSubDir{ch-Preliminaries}
\SetExpSubDir{ch-Preliminaries}

\chapter{Preliminaries}
\label{ch:preliminaries}
\vspace{2em}

We will discuss the technical details of top responsive games,
various stability concepts, mutuality, and PAC stability in this chapter.

\section{Hedonic Game}
A \textit{hedonic game} is a pair $(N, P)$, where $N$ is a finite set of players
$\{1, \cdots, n\}$ and $P$ is a \textit{preference profile} consisting of
preference relations $\succeq_i$ for every player $i \in N$:
$P = (\succeq_1, \cdots, \succeq_n)$.
A \textit{preference relation} $\succeq_i$ is a reflective, complete, and
transitive binary relation on $\mathcal{N}_i$, where $\mathcal{N}_i$ is the set
of all non-empty subset of $N$ that includes player $i$;
i.e., $\mathcal{N}_i = \{S \subseteq N: i \in S, S \neq \emptyset \}$

Consider a parliament; who are "natural allies"? One way to model this is by
assuming that each member has a preference relation over various groups they
wish to be part of; our objective is to identify partitions
(also known as \textit{coalition structures}) that satisfy certain desirable
properties, such as stability.

\subsection{Hedonic Preference Models}
We are learning preferences from data, but still need to make certain
assumptions about a player's utility model.
We now describe the two classes of preference models our study is built upon.

\paragraph{Top Responsive Preferences}
The intuition behind top responsiveness is that every player derives their
utility from the most preferred subset players of the coalition they belong.
If two coalitions, with one containing the other, yield the same utility for
a player, the tie is broken in favor of the smaller coalition.

In the context of our dataset, it captures the following idea:
politicians care about whose votes they stand with;
they want to vote with other politicians they like.
If they can manage to pass a bill with fewer members involved,
it is more efficient therefore more preferable.

A player $i$'s most preferred sets of coalitions is called \textit{choice sets}:
$\Ch(i, S) = \{S' \subseteq S: (i \in S') \wedge (S' \succeq_i S'' \forall S'' \subseteq S)\}$.
The unique choice set in $\Ch(i, S)$ is denoted as $\ch(i, S)$.
A {\em top responsive} preference profile requires that for any player $i \in N$,
and any coalition that may contain player $i$: $S, T \in \mathcal{N}_i$:
\begin{enumerate}
  \item $|\Ch(i, S)| = 1$.
  \item if $\ch(i, S) \succ_i \ch(i, T)$ then $S \succ_i T$
  \item if $\ch(i, S) = \ch(i, T)$ and $S \subset T$ then $S \succ_i T$
\end{enumerate}

The description of a generic top responsive preference profile is exponentially
large in the number of players, as each player needs to rank all possible
coalitions they may belong to.
In our implementation and experiments in Chapter~\ref{ch:hedonic}, we will
explore top responsive models in both its generic form
(Section~\ref{subsec:handcrafted_value_function}) and specific sub-classes of
this preference model (Section~\ref{subsec:appreciation_of_friends},
Section~\ref{subsec:aversion_to_enemies}).

\paragraph{Boolean Preferences}
TODO

\subsection{Stability Concepts} \label{subsec:stability}
We are particularly interested in models that produce partitions that satisfy
strategic considerations; specifically, we focus our attention on stability
concepts that capture the idea that no group of players can be better off by
leaving and forming their own coalition.
Group based stability notions are stronger than those based on individual deviation,
such as Nash stability, because individual based stability concepts do not account
for the disutility of the group an individual player is joining or leaving.

\paragraph{Core \& Strict Core}
A coalition $S \subseteq N$ \textit{strongly blocks} a coalition structure $\pi$
if every player $i \in S$ strictly prefers $S$ over its current coalition $\pi(i)$;
a coalition structure $\pi$ is \textit{core stable} when there is no strongly
blocking coalition $S$.
A coalition $S \subseteq N$ \textit{weakly blocks} a coalition structure $\pi$
if every player $i \in S$ weakly prefers $S$ over its current coalition $\pi(i)$
and there exists at least one player $j \in S$ who strictly prefers $S$
over $\pi(j)$; a coalition structure is strictly core stable when there is no
weakly blocking coalition.

\paragraph{Strong Nash \& Strict Strong Nash}
Next we cover two solution concepts with even stronger notions of stability
based on group deviation.
If a partition $\pi' \neq \pi$ exists with movement of players $S \subseteq N$
and $S \neq \emptyset$ (denoted as $\pi \xrightarrow{S} \pi'$), where
$\forall i \in S$, $\pi'(i) \succ_i \pi(i)$,
and $\forall j \in N\text{\textbackslash}S$, $\pi'(j) = \pi(j)$,
then $S$ strongly Nash blocks $\pi$.
A partition that admits no strongly Nash blocking set $S \subseteq N$ is said
to be strong Nash stable (SNS).
A non-empty set of players $S \subseteq N$ weakly Nash blocks $\pi$ if
$\forall i \in S$, $\pi'(i) \succeq_i \pi(i)$ and $\exists j \in S$,
$\pi'(j) \succ_j \pi(j)$.
A partition that admits no weakly Nash blocking set $S \subseteq N$ is said to
be strict strong Nash stable (SSNS).

\paragraph{}
A top responsive preference profile guarantees the existence of a strict core partition.
Moreover, top responsiveness also warrants a strict strong Nash stable partition
if all players preference relations are \textit{mutual} ––– for all $i, j \in N$,
for all coalition that contains $i$ and $j$:
$S \in \mathcal{N}_i \cap \mathcal{N}_j$, $i \in ch(j, S)$ if and only if
$j \in ch(i, S)$.
This means if we can model empirical data with mutual preferences,
we can achieve the strongest notion of group based notion of stability.


\subsection{PAC Learning and PAC Stability}
Probably Approximately Correct (PAC) learning is the canonical framework for provable probabilistic approximations to functions. In this framework, a learner receives samples and selects a function from a class of possible functions. The selected function is called the \textit{hypothesis}, which should be likely to predict new samples from the same distribution. A good probabilistic approximation means that with probability of at least $1 - \delta$, the selected function's output has an average error less than or equal to $\varepsilon$, where $0 < \varepsilon, \delta < 1$. A hypothesis class is efficiently PAC learnable if such a good probabilistic approximation can be produced by some algorithm that has both running time and input sample size be polynomial in $n$, $\frac{1}{\varepsilon}$, and $\log{\frac{1}{\delta}}$.

\citename{ijcai2017-380} define that a class of hedonic games is PAC stablizable if there exists an algorithm that produces a partition $\pi$ such that $\Pr_{S\sim D}[\text{S core blocks } \pi] < \varepsilon$; both the number of samples required by the algorithm to provide this PAC guarantee, and the running time of producing a consistent solution are polynomial in $n$, $\frac{1}{\varepsilon}$, and $\log{\frac{1}{\delta}}$. In the same paper, \citename{ijcai2017-380} also showed that top responsive games are efficiently PAC stablizable even though they are not PAC learnable.
A PAC stable partition can be computed with Algorithm~\ref{alg:pac_top_covering} which we will describe in detail in Chapter~\ref{ch:hedonic} Section~\ref{sec:top_responsive_game} Subsection~\ref{subsec:algorithms}.
