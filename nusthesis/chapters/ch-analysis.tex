\SetPicSubDir{ch-Analysis}
\SetExpSubDir{ch-Analysis}

\chapter{Experiment Analysis}
\label{ch:analysis}
\vspace{2em}

\section{Results \& Discussion}
\label{sec:results_and_discussion}

\subsection{Variability among PAC Partitions}
\label{subsec:variability_among_pac_partitions}

Before we can meaningfully aggregate 50 partition results from every PAC model
we need to assess how different these 50 partitions are from each other.
If they are vastly different, it would be evidence that there is error in the
PAC algorithm that produces these partitions; otherwise it would act as a sanity
check on the correctness of our implementation of each corresponding algorithm.

Table~\ref{analysis:table:pac_num_coalitions} presents the mean and standard
deviation of the number of coalitions across 50 runs for every PAC model.
Coefficient of variation (CV) is the ratio of the standard deviation to the mean.
Note that all the PAC models have CV values well below 1, so there is very low
variation in terms of partition size across 50 runs for every PAC model.

\begin{table}[h!]
\centering
\begin{tabular}{|c|c|c|c|}
\hline
       & Mean Partition Size & Standard Deviation & CV \\ \hline
Value Function & 87 & 0 & 0 \\
General Friends & 13 & 1.29 & 0.10  \\
Selective Friends & 20 & 1.83 & 0.09  \\
Selective Enemies & 10 & 0 & 0 \\
General Enemies & 34 & 0.84 & 0.02 \\
Boolean & 31 & 0.46 & 0.02  \\
\hline
\end{tabular}
\caption{PAC Model Partition Size Statistics}
\label{analysis:table:pac_num_coalitions}
\end{table}

Recall that Adjusted Mutual Information (AMI) measures how similar two
partitions are and is very sensitive in detecting very different partitions (
Chapter~\ref{subsec:information_theoretic_measures}).
We calculate the pairwise AMIs among the 50 partitions for each PAC model;
the resulting statistics are presented in
Table~\ref{analysis:table:pac_pairwise_amis}.
We observe even the smallest pairwise AMI for every PAC model $>> 0$ and all CVs
$<< 1$, from which we can conclude that, beyond partition size, the content
of every partition is not dissimilar from each other, consistently across 50
partitions.

\begin{table}[h!]
\centering
\begin{tabular}{|c|c|c|c|}
\hline
       & Minimum Pairwise AMI & Mean Pairwise AMI & CV \\ \hline
Value Function & 1 & 1 & 0 \\
General Friends & 0.6 & 0.78 & 0.09  \\
Selective Friends & 0.66 & 0.84 & 0.08  \\
Selective Enemies & 0.97 & 0.99 & 0.01 \\
General Enemies & 0.93 & 0.97 & 0.01 \\
Boolean & 0.83 & 0.93 & 0.06  \\
\hline
\end{tabular}
\caption{PAC Model Partition Pairwise AMI Statistics}
\label{analysis:table:pac_pairwise_amis}
\end{table}

Now that we have established there is little variation within every set of 50
PAC partitions, as described in Section~\ref{sec:experiment_design}, we select
the ``centroid'' of each set as the representative of its corresponding PAC model.
Since all the partitions are reasonably close to each other, it does not
matter either we use AMI or VI for selecting the centroid.
Given that VI is a distance measure therefore more intuitive to work with in
this case, we choose each representative partition as the one with the smallest
sum of VIs to the rest 49 partitions.

\subsection{Model Partitions Compared to Partition by Party Affiliation}
\label{subsec:partition_comparisons}

Using party affiliations as the ground truth partition allows us to use
quantitative information theoretic measures to quickly compare across all
models.
We can then pick out those models with partitions that are drastically
different from the party affiliations and delve into the reason behind such
differences.
Recall that Adjusted Mutual Information (AMI) is suitable in performing
this task (Section~\ref{subsec:information_theoretic_measures}).
For partition produced by every model described in Chapter~\ref{ch:hedonic} and
Chapter~\ref{ch:comparison}, we calculate the (AMI) between it and the partition
by party affiliation.
The results are visualized in Figure~\ref{Analysis:fig:ami}.

\begin{figure}[!t]
  \centering
  \includegraphics[width=\linewidth]{\Pic{png}{ami}}
  \vspace{\BeforeCaptionVSpace}
  \caption{Adjusted Mutual Information (AMI) between model partition and party affiliations}
  \label{Analysis:fig:ami}
\end{figure}

The full-information friends model scored an AMI of $0.000$ as the partition is
simply the grand coalition.
As discussed in Section~\ref{subsec:appreciation_of_friends}, under both general
friends and selective friends model, a player prefer coalitions with more
friends --- it means that the more generous the definition of ``friends'' the
more likely we can expect to observe the grand coalition as the core stable
coalition structure.
This intuition is verified by the fact that the AMI for the full-information
selective friends model is much higher (0.29) compared to that of the general
friends model.

It also demonstrates a major drawback of the full-information versions of the
appreciation of friends model --- they are sensitive to the definition of
``friends''; a slight change from the narrow disagreement of general friends
to the broad disagreement of selective friends produces two very different, yet
both core stable coalition structures with respect to their own definition of
friends.
In our friends models, a player effectively considers someone a friend when
this person agrees with their votes over $50\%$ of the time.
We can easily adjust this threshold up or down to produce a spectrum of
definitions of friends.
Without peeking at the ground truth coalition, it is unclear how one can pick
the ``best'' definition.

Observe that the AMIs for the PAC versions of the two friends models are much
closer: $\AMI_{\text{PAC friends}} = 0.292,
\AMI_{\text{PAC selective friends}} = 0.287$.
This is evidence that through sampling, PAC models dampen their sensitivity to
the definition of friends.

\paragraph{} The second worst partition, according to AMI with the party affiliation
partition, is the full-information Boolean model ($\AMI_{\text{Boolean}} = 0.018$).

% TODO: how not to duplicate figure? Inline reference of figure from appendix?
\begin{figure}[!t]
  \centering
  \includegraphics[width=\linewidth]{\Pic{png}{boolean}}
  \vspace{\BeforeCaptionVSpace}
  \caption{Coalition structure produced by the Boolean model}
  \label{Analysis:fig:boolean_partition}
\end{figure}

Upon close inspection of the partition produced by the full-information Boolean
model compared to party affiliations (Figure~\ref{Analysis:fig:boolean_partition},
we notice that the Boolean partition contains one large coalition made of 115
players of various parties from both sides of the aisle (Coalition 1),
and one two-member coalition made of two players from the Joint List (Coalition 2);
all other coalitions are singletons.
This is a result of a combination of the preference model and the core finding
algorithm for Boolean hedonic games:
Since the algorithm looks for the largest coalition in which everyone is
satisfied, the largest bill with cross-party support/disapproval forms the
largest coalition.
Once this largest coalition is formed, we remove any preference with players
in this coalition, effectively removing most satisfactory coalitions for
remaining players.
Since a player is indifferent among unsatisfactory coalitions, we default to
placing them in coalitions containing themselves, resulting in most remaining
players in singleton coalitions.

\paragraph{} The third worst partition, which is also the worst PAC partition
among all PAC models, is that of the PAC value function model
($\AMI_{\text{PAC Value Fuction}} = 0.144$).
As illustrated in Figure~\ref{Analysis:fig:value_function_pac_partition}, the
PAC value function partition has one large coalition consisting of around 60
members and the rest are all singleton coalitions.
All members of the only large coalition are from right wing parties that
together form the thirty-fourth government of Israel.
Observe that the full-information version of the value function model also
produces a similar government coalition (Coalition 2 of
Figure~\ref{Analysis:fig:value_function_partition}).
Apart from the government coalition, the full-information value function model
also produces a opposition coalition (Coalition 1) made of left wing party
members and a few members from a center-right/right-wing party
(Yisrael Beiteinu).
The full-information value function model suggests that most parliament members
are voting according to their party's political position (i.e. left wing vs.
right wing, government vs. opposition).
This matches our understanding of the Israel political system described in
Section~\ref{sec:data}.
However, lumping six out of eight Yisrael Beiteinu members together with other
opposition/left-wing party members in Coalition 1 is not in line with reality.
One possible reason why we observe the opposition coalition dissolve while the
government coalition remains in the PAC version of the model is that the
government party members has a dominating advantage in the Knesset --- since our
value function (Function~\ref{eq:value_function} values a winning majority over
a losing coalition, the parties forming the government more frequently pass
or oppose bills successfully than the opposition, therefore they are more
likely getting picked by the PAC sampling process.
This also highlights the limitations of our handcrafted value function.

\begin{figure}[!t]
  \centering
  \begin{minipage}{0.48\linewidth}
    \centering
    \includegraphics[width=\linewidth]{\Pic{png}{value_function}}
    \vspace{\BeforeCaptionVSpace}
    \caption{Coalition structure produced by the value function model}
    \label{Analysis:fig:value_function_partition}
  \end{minipage}\hfill
  \hspace{.03\linewidth}
  \begin{minipage}{0.48\linewidth}
    \centering
    \includegraphics[width=\linewidth]{\Pic{png}{value_function_pac}}
    \vspace{\BeforeCaptionVSpace}
    \caption{Coalition structure produced by the PAC value function model}
    \label{Analysis:fig:value_function_pac_partition}
  \end{minipage}
\end{figure}


\begin{figure}[!t]
  \centering
  \includegraphics[width=\linewidth]{\Pic{png}{vi}}
  \vspace{\BeforeCaptionVSpace}
  \caption{Variation of Information (VI) between model partition and party affiliations}
  \label{Analysis:fig:vi}
\end{figure}

\subsection{Partition Visualization}
\label{subsec:partition_visualization}

We color the political parties according to their political position --- 
right wing parties in red-ish hues and left wing parties in blue-ish hues.

% TODO: add parties visualized in color

We visualize the partition assignments of every parliament member under every model
in Section~\ref{sec:detailed_partitions}.
