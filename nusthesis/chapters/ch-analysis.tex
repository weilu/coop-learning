\SetPicSubDir{ch-Analysis}
\SetExpSubDir{ch-Analysis}

\chapter{Results \& Discussion}
\label{ch:analysis}

In this chapter, we first examine and aggregate our experimental results
(\autoref{sec:variability_among_pac_partitions}).
In the process of reasoning about our experimental results, we realize the
importance of good data visualization --- even after aggregation, we have over
16 model partitions, each with over a hundred parliament members to present;
as a result, we require a succinct visualization describing partition outputs
from various models.
The visualization should not only be intuitively understandable, but also aid
our qualitative analysis (\autoref{sec:partition_visualization}).
Following the presentation of results\footnote{In the interest of space we
place the presentation of results in \autoref{append:viz}}, we conduct
quantitative analysis by comparing model clusters against party affiliations
using information theoretic measures (\autoref{sec:quantitative_analysis}).
After narrowing down the number of models using quantitative analysis, we
proceed to comparing and discussing the validity of and insights revealed by
the selected models (\autoref{sec:qualitative_analysis}).


\section{Variability among PAC Partitions}
\label{sec:variability_among_pac_partitions}
Recall that, by design, we repeat the experiment 50 times for each PAC model
in order to evaluate the robustness of the outputs (\autoref{sec:experiment_design}).
In this section we conduct the evaluation by assessing how much do these 50
partitions differ.
If they are vastly different, this serves as evidence that there is issue in the
PAC algorithm that produces these partitions; if they are relatively consistent,
we can then proceed to aggregate each set of 50 partition results for every PAC
model.

\autoref{Analysis:table:pac_num_coalitions} presents the mean and standard
deviation of the number of coalitions across 50 runs for every PAC model.
The coefficient of variation (CV) is the ratio of the standard deviation to the mean; lower values generally indicate low variation across samples.
Note that all the PAC models have CV values well below 1, so there is very low
variation in terms of partition size across 50 runs for every PAC model.

\begin{table}[ht]
\centering
\begin{tabular}{|c|c|c|c|}
\hline
  Model & Mean Partition Size & Standard Deviation & CV \\ \hline
Value Function & 87 & 0 & 0 \\
General Friends & 13 & 1.29 & 0.10  \\
Selective Friends & 20 & 1.83 & 0.09  \\
Selective Enemies & 10 & 0 & 0 \\
General Enemies & 34 & 0.84 & 0.02 \\
Boolean & 31 & 0.46 & 0.02  \\
\hline
\end{tabular}
\caption{PAC Model Partition Size Statistics over 50 Runs per Model}
\label{Analysis:table:pac_num_coalitions}
\end{table}

Recall that Adjusted Mutual Information (AMI) measures how similar two
partitions are and is very sensitive in detecting very different partitions
(\autoref{subsec:information_theoretic_measures}).
We calculate the pairwise AMIs among the 50 partitions for each PAC model;
the resulting statistics are presented in \autoref{Analysis:table:pac_pairwise_amis}.
We observe even the smallest pairwise AMI for every PAC model $>> 0$ and all CVs
$<< 1$, from which we can conclude that, beyond partition size, partitions
generated by our methods are relatively similar across different random samples.

\begin{table}[ht]
\centering
\begin{tabular}{|c|c|c|c|}
\hline
  Model & Minimum Pairwise AMI & Mean Pairwise AMI & CV \\ \hline
Value Function & 1 & 1 & 0 \\
General Friends & 0.6 & 0.78 & 0.09  \\
Selective Friends & 0.66 & 0.84 & 0.08  \\
Selective Enemies & 0.97 & 0.99 & 0.01 \\
General Enemies & 0.93 & 0.97 & 0.01 \\
Boolean & 0.83 & 0.93 & 0.06  \\
\hline
\end{tabular}
\caption{PAC Model Partition Pairwise AMI Statistics over 50 Runs per Model}
\label{Analysis:table:pac_pairwise_amis}
\end{table}

Having established low variation within every set of 50
PAC partitions, as described in \autoref{sec:experiment_design}, we select
the ``centroid'' of each set as the representative of its corresponding PAC model.
Since all the partitions are reasonably close to each other, it does not
matter either we use AMI or VI for selecting the centroid.
Given that VI is a distance measure, and is thus more intuitive to work with in
this case, we choose each representative partition as the one with the smallest
sum of VIs to the other 49 partitions.

\section{Partition Visualization}
\label{sec:partition_visualization}

Before delving into the model outcome comparisons and analysis, we briefly
describe the visualizations used to depict coalition assignments of parliament members, and to compare against their party affiliations which are
used as the ground truth in our quantitative analysis.

\begin{figure}[h]
  \centering
  \includegraphics[width=.6\linewidth]{\Pic{png}{k_means_2}}
  \caption{Type 1 Sankey diagram example: coalition structure produced by the $k$-means model ($k=2$)}
  \label{Analysis:fig:k_means_2_partition}
\end{figure}

We choose \textit{Sankey Diagrams} as our main visualization method.
It allows intuitive comparison of two different partitions through ``flows''.
Sankey diagrams are a type of flow diagram which is often used to represent
change of states in a system \cite{doi:10.1111/j.1530-9290.2008.00004.x}.
In our case, on the left we have the ground truth state which is the partition
by party affiliation; on the right is one of our model partitions.
We produce two types of Sankey diagrams:

\begin{enumerate}
  \item Without individual members, the left party partition state is directly
    linked to the right model partition (\autoref{sec:partitions}).
  \item With every parliament member as nodes in the middle, thus the flow links
    from left to middle represent their party affiliations and the flow links
    from middle to right represent our model assignments
    (\autoref{sec:detailed_partitions}).
\end{enumerate}

\autoref{Analysis:fig:k_means_2_partition} is an example of the first type
of Sankey diagram used to visualize one of the ML models, $k$-means ($k=2$).
For the corresponding detailed view (Type 2) of the same model, see
\autoref{fig:k_means_2_partition_detailed}.

The second type provides detailed view for when we want to delve into exactly
which parliament member is grouped with which other members; it has the
downside of being overly verbose.
The first type simplifies the presentation, which is good for visual comparison
of the ground truth partition against any given model partition.
We provide an interactive version of the first type of Sankey diagrams
\footnote{available at \url{https://weilu.github.io/coop-learning/}}; this
interface allows readers to observe the exact assignment of parliament members
to coalitions under various algorithms, measurement tools, and distance metrics.

We color the political parties according to their political position ---
right wing parties in reddish hues and left wing parties in blueish hues.
This way, we can quickly judge if a model successfully separates the government
(right wing) parties from the opposition (central to left wing) parties.

\section{Quantitative Analysis}
\label{sec:quantitative_analysis}

Using party affiliations as the ground truth partition, we use
quantitative information theoretic measures to quickly compare our models.
We can then identify models whose output partitions are drastically
different from the party affiliations, and delve into the reason behind such
differences.
Recall that Adjusted Mutual Information (AMI) is suitable for performing
this task (\autoref{subsec:information_theoretic_measures}).
Given a partition produced by a model described in \autoref{ch:hedonic}
and~\ref{ch:comparison}, we calculate the AMI between it and the ground truth party
affiliation partition.
The results are visualized in \autoref{Analysis:fig:ami}.

\begin{figure}[ht]
  \centering
  \includegraphics[width=\linewidth]{\Pic{png}{ami}}
  \caption{Adjusted Mutual Information (AMI) between model partition and party affiliations}
  \label{Analysis:fig:ami}
\end{figure}

\paragraph{Worst Performing Model Under AMI}
The full-information friends model scored an AMI of $0.000$ as the partition is
simply the grand coalition.
As discussed in \autoref{subsec:appreciation_of_friends}, under both general
friends and selective friends model, players prefer coalitions with more
friends --- this means that the more generous the definition of ``friends'', the
more likely we can expect to observe the grand coalition as the core stable
coalition structure.
This intuition is verified by the fact that the AMI for the full-information
selective friends model is much higher (0.29) compared to that of the general
friends model.

It also demonstrates a major drawback of the full-information versions of the
appreciation of friends model --- they are sensitive to the definition of
``friends''; a slight change from the narrow disagreement of general friends
to the broad disagreement of selective friends produces two very different, yet
both core stable coalition structures with respect to their own definition of
friends.
In our friends models, a player considers someone a friend when this person
agrees with their votes over $50\%$ of the time.
We can easily adjust this threshold to produce a spectrum of definitions of
friends.
Under the full-information model, it is unclear how one can pick the ``best''
threshold without access to prior ground-truth knowledge.

Observe that the AMIs for the PAC versions of the two friends models are much
closer: $\AMI_{\text{PAC friends}} = 0.292,
\AMI_{\text{PAC selective friends}} = 0.287$.
This is evidence that through sampling, PAC models dampen their sensitivity to
the definition of friends.

% TODO: how not to duplicate figure? Inline reference of figure from appendix?
\begin{figure}[ht]
  \centering
  \includegraphics[width=\linewidth]{\Pic{png}{boolean}}
  \caption{Coalition structure produced by the Boolean model}
  \label{Analysis:fig:boolean_partition}
\end{figure}

\paragraph{Second Worst Performing Model Under AMI} 
The second worst partition, according to AMI with the party affiliation
partition, is the full-information Boolean model ($\AMI_{\text{Boolean}} = 0.018$).
Upon closer inspection of the partition produced by the full-information Boolean
model compared to party affiliations (\autoref{Analysis:fig:boolean_partition}),
we notice that the Boolean game partition contains one large coalition made of 115
players of various parties from both sides of the aisle (Coalition 1),
and one two-member coalition made of two players from the Joint List (Coalition 2);
all other coalitions are singletons.
This is a result of a combination of the preference model and the core finding
algorithm for Boolean hedonic games:
Since the algorithm looks for the largest coalition in which everyone is
satisfied, the largest bill with cross-party support/disapproval forms the
largest coalition.
Once this largest coalition is formed, we remove any preference with players
in this coalition, effectively removing most satisfactory coalitions for
remaining players.
Since a player is indifferent among unsatisfactory coalitions, we default to
placing them in coalitions containing themselves, resulting in most remaining
players in singleton coalitions.

\paragraph{Third Worst Performing Model Under AMI}
The third worst partition, which is also the worst PAC partition
among all PAC models, is that of the PAC value function model
($\AMI_{\text{PAC Value Fuction}} = 0.144$).
As illustrated in \autoref{Analysis:fig:value_function_pac_partition}, the
PAC value function partition has one large coalition consisting of around 60
members and the rest are all singleton coalitions.
All members of the large coalition are right wing parties that formed the 34th
government of Israel.
Observe that the full-information version of the value function model also
produces a similar government coalition (Coalition 2 of
\autoref{Analysis:fig:value_function_partition}).
Apart from the government coalition, the full-information value function model
also produces an opposition coalition (Coalition 1) made of left wing party
members and a few members from a center-right/right-wing party
(Yisrael Beiteinu).
The full-information value function model suggests that most parliament members
are voting according to their party's political position (i.e. left wing vs.
right wing, government vs. opposition).
This matches our understanding of the Israeli political system described in
\autoref{sec:data}.
However, lumping six out of eight Yisrael Beiteinu members together with other
opposition/left-wing party members in Coalition 1 is not in line with reality.
One possible reason why we observe the opposition coalition dissolve while the
government coalition remains in the PAC version of the model is that the
government party members has a dominating advantage in the Knesset --- since our
value function (Function~\ref{eq:value_function} values a winning majority over
a losing coalition, the parties forming the government more frequently pass
or oppose bills successfully than the opposition, therefore they are more
likely getting picked by the PAC sampling process.
This also highlights the limitations of our handcrafted value function.

\begin{figure}[ht]
  \centering
  \begin{minipage}{0.48\linewidth}
    \centering
    \includegraphics[width=\linewidth]{\Pic{png}{value_function}}
    \caption{Coalition structure produced by the value function model}
    \label{Analysis:fig:value_function_partition}
  \end{minipage}\hfill
  \hspace{.03\linewidth}
  \begin{minipage}{0.48\linewidth}
    \centering
    \includegraphics[width=\linewidth]{\Pic{png}{value_function_pac}}
    \caption{Coalition structure produced by the PAC value function model}
    \label{Analysis:fig:value_function_pac_partition}
  \end{minipage}
\end{figure}

\paragraph{Fourth Worst Performing Model Under AMI}
The partition produced by SBM with normal edge weight distribution
also performed poorly when measured against party affiliations
($\AMI_{\text{SBM normal}} = 0.192$).
It contains 32 coalitions, all of which are small in size (2-9 players in each
coalition).
As visualized in \autoref{fig:sbm_normal_partition}, though it mostly
clusters left wing party members with left wing party members, and right wing
party members with other right wing party members, it fails to distinguish the
government from the opposition.
The edge weight distribution's departure from the normal distribution
(as depicted in \autoref{Comparison:fig:sbm_edge_weight_neg_hist}) could
have contributed to the poor clustering results.

\begin{figure}[ht]
  \centering
  \includegraphics[width=\linewidth]{\Pic{png}{vi}}
  \caption{Variation of Information (VI) between model partition and party affiliations}
  \label{Analysis:fig:vi}
\end{figure}

\paragraph{Model Selection Using AMI \& VI}
We now cross check the four worst performing models discussed above using the
Variation of Information (VI) measurements (\autoref{Analysis:fig:vi}).
The rankings of the four models are summarized in
\autoref{Analysis:table:ami_vi_worst_rankings}.
Both measurements agree that Boolean, PAC value function, and SBM Normal compare poorly
to the party affiliations, however they disagree on the general friends
model.
If we focus our attention on comparing the general vs. selective friends, we
observe that both measures deem the selective friends versions outperform
(or perform equally as) their general counterpart.
Likewise, selective enemies models outperform their general counterpart in
both AMI and VI measurements. \footnote{Recall that selective enemies models
use narrow disagreement while selective friends models use broad disagreement;
they essentially use different definitions of ``friends'' despite sharing
similar names}.
With in selective friends and selective enemies models, the PAC versions
score very similarly to their full-information versions.
As such, we will focus our comparisons on the PAC models with selective friends
, selective enemies and Boolean, the 2-group and 10-group k-means outcomes
(selected using the average silhouette method and elbow method respectively,
see \autoref{sec:k_means_clustering}),
and the SBM geometric for the community models.

\begin{table}[ht]
\centering
\begin{tabular}{|c|c|c|c|}
\hline
       & AMI Ranking & VI Ranking \\ \hline
General Friends & Worst & 8th Worst \\
Boolean & 2nd Worst & 2nd Worst \\
Value Function (PAC) & 3rd Worst & 5th Worst \\
SBM Normal & 4th Worst & Worst \\
\hline
\end{tabular}
\caption{VI and AMI Rankings of selected models}
\label{Analysis:table:ami_vi_worst_rankings}
\end{table}


\section{Qualitative Analysis}
\label{sec:qualitative_analysis}

One technical note before we delve into qualitative analysis of the selected
models: due to Knesset members resigning mid term, some Knesset members only participated in some votes, and are therefore often clustered separately.
Moreover, some government ministers which are also Knesset members have a low
rate of participation in the votes, and therefore similarly their cluster
assignment may be skewed as a result.

\paragraph{Coherence}
Do models separate coalition and opposition parties? Models that are able to identify these lines are referred to as coherent.

Under this criterion, all three selected PAC models: selective friends,
selective enemies, and Boolean are the only ones which did not include several
groups involving many members of different ideological hue.
The $k$-means (with $k=10$) model, and the community model (SBM geometric) tend to
join together members with fewer votes, regardless of actual voting patterns.
%TODO: insert viz by vote count%
Examples of such cross-ideological infrequent-voter groups are: Coalition 2
of the $k$-means ($k=10$) model, Coalition 8 of the SBM geometric model.
The $k$-means ($k=2$) model partition includes five members of the right wing
parties in the opposition group (Coalition 2), out of which only one such
assignment makes sense --- Orly Levy left the right/center-right Yisrael
Beiteinu and formed her own party `Gesher', which ran together with the center
left Labor Party in the 2019 Knesset elections
\footnote{\url{https://en.wikipedia.org/wiki/Orly_Levy}}; so it is reasonable to
see her grouped together with the opposition group.

Overall, the PAC models fare much better in terms of coherence:
the PAC selective friends model has only two cross-ideological groups ---
Coalition 2 involves Orly Levy which, as previously mentioned, has a reasonable
explanation; Coalition 5 is a small coalition combining two low-attendance
members, one on the right edge of the left wing (Daniel Atar with 708 effective
votes\footnote{Recall that we only consider ``for'' and ``against'' as
effective votes}), and another, who switched between coalition and opposition
(Avigdor Lieberman with 725 effective votes).
The PAC selective enemies model includes two right wing party members in the
opposition group (Orly Levy and Avigdor Lieberman), both of which we have
explained above.
The PAC Boolean model only has one cross-ideological group involving Orly Levy,
the member who changed parties.

\paragraph{Overall structure}
Both PAC models and the $k$-means ($k=2$) are able to distinguish between the
main government and opposition groups, to which most
of their respective members were attached.
This stresses the fact that despite their differences, especially the opposition
parties which do not have a centralized binding instructions, still tend to
vote together due to their relative ideological cohesion.
The community based SBM geometric model and $k$-means ($k=10$) model, however,
did not really pick up on the existence of the relatively coherent group of
government and opposition Knesset members.

Most other groups were created based on attendance, with the PAC models more
clearly creating a few singleton coalitions for less present members, while
clustering ministers, who tend to miss votes, together.
Ministers are clustered in two different coalitions by PAC selective friends:
Coalition 3 and Coalition 4, and are clustered altogether in Coalition 4 by PAC selective
enemies model.
PAC Boolean model notably creates singletons for most low attendance members.
Both the $k$-means ($k=10$) and the SBM geometric clusters tended to remove
more ministers from the overall main government group, and divide them into
more separate coalitions.

\paragraph{Party Sub-groups}
None of the models were able to distinguish between the subgroups within
parties, apart from identifying a significant change (e.g., a member switches
parties).
We believe this is more of an indication of the process of ideological
coherence happening in Israeli parties.
Parties that have been sharing similar ideologies have recently joined together to run as a single ticket do not have many opportunities to flaunt their differences.
Thus, even for a party, such as the Joint List, that did not run together in
the following election, it does not seem that their Knesset members voted
differently.
Indeed, several of their Knesset members, despite being from different parties
(and running separately in the following election) have not been assigned to
different coalitions by any model.
As noted above, we believe this tells us a meaningful statement on the real,
day to day, ideological difference between them, and is not a failure of the
algorithms.

