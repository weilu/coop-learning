\SetPicSubDir{ch-Experiment}
\SetExpSubDir{ch-Experiment}

\chapter{Experimental Methodology}
\label{ch:experiment}

\section{Data}
\label{sec:data}

The Israeli political system consists of multiple parties, partially due to its
use of a proportional voting system, and its diverse political landscape.
The Israeli Knesset is the national legislative branch of the Israeli government.
We limit our study to the 20th Knesset (2015-2019), which is the last
Knesset preceding the current one.
In this Knesset there are ten parties, however, its political landscape is far more nuanced.

%TODO: citations
Some parties are `artificial' in the sense that they are the result of a merge
of smaller political factions.
For example, the Jewish Home party comprises of three conservative religious
parties which agreed to run together.
Similarly, the Joint List (as its name suggests) comprises of all Arab National
parties.
In addition, party leadership and member election methods vary widely.
For example, the Likud (the leadership party in the 20th Knesset) and Meretz
(a socialist progressive party) hold primary elections to elect their leader
and party members; other parties, such as Yesh Atid (a centrist party) do not
hold elections but rather have the party leader elect its members.
See \autoref{table:knesset_party_members} for a complete list of the 20th Knesset
members, their party affiliations and ideologies.

While in the past, this preponderance of parties led to a multi-dimensional
party system, leading to unexpected coalitions and combinations,
in the past few decades Israeli parties can be, in general,
ordered along a single right-left axis, which has to do mainly with
the parties' approach to the Israeli-Palestinian conflict.
This simplifies the considerations we need to take into account when analyzing
the outcomes, and allows for easier comparison between the different models.
Moreover, it allows us to more easily discern
unexpected coalition structures that need to be explained.
There is also relative ideological cohesion between coalition parties, which
form the government, as well as between opposition parties.

No Israeli party has ever been elected with a majority of Knesset seats,
which means all governments are made of coalitions.
In the past few years (including the Knesset we investigate)
governments have tightened their grip over coalition parties by having a government committee issue generally mandatory
voting instructions for every proposed bill.
The opposition, of course, is under no such control,
but due to their relative ideologies, there is significant agreement.

The Knesset website provides data access through Open Data Protocol (OData)
on all its parliament members, laws, and member votes on every law.
The Kenesset has 120 seats, but the 20th Knesset has 147 parliament members
due to some Knesset members resigning or joining mid-term.
We download all 147 parliament members' information including name,
party affiliation and their votes for all 7515 bills deliberated.
A vote can take on one of the following values
0 (vote canceled), 1 (vote for), 2 (vote against),
3 (abstained), 4 (did not attend).

\subsection{Data Preprocessing}
\label{subsec:data_preprocessing}

During data processing, we notice that more than half of all the vote values
are missing despite the presence of value 4 for ``did not attend''.
It raises questions regarding the data quality.
We discover another API endpoint which provides summary information for
every bill, including total number of ``for'' votes, ``against'' votes,
``abstained'' votes, and if the bill is accepted.
We download bill summary data set for the 20th Knesset and use it
to check against the tallied vote numbers from the individual vote data.
We ended up discovering more inconsistencies:

\begin{itemize}
  \item Total bills checked: 7513,
  \item Missing bills in the summary dataset: 2,
  \item Number of bills with pass/reject status inconsistent: 26,
  \item Number of bills with for count inconsistent : 273,
  \item Number of bills with against count inconsistent : 402,
  \item Number of bills with abstain count inconsistent : 2
\end{itemize}

We reached out to the Knesset data management team regarding the data inconsistency between
the two API endpoints, as well as the missing value issue.
They stated that the inconsistency
is likely a result of manual vote entries instead of electronic votes recorded\footnote{This practice violates the parliament voting protocol, but was, nevertheless, accepted by the Speaker of the Knesset.}.
Such manual vote entries are captured by the bill summary dataset but not individual members' vote dataset.

For the purpose of our research, we use the dataset with individual member
votes on every bill since it has the level of granularity required for modeling.
We remove the 26 bills whose pass/reject status is inconsistent between
the summary and individual data sets, since it is most likely that the
individual data is wrong.

% Maybe include data summary tables?
% total bills, total parliament members
% total for, against, abstained, did not attend, missing votes
% per bill: min, max, average for, against, abstained, did not attend, missing votes
% per member: min, max, average for, against, abstained, did not attend, missing votes

\section{Experiment Design}
\label{sec:experiment_design}

Recall our first research questions: Can we use hedonic games to model
real-world collaborative activities?
We see in past works \cite{Vainsencher2011BundleSB, Balcan2012}, to demonstrate
the power of game theoretic models in describing real-world scenarios, it
usually involves the following two steps:

\begin{enumerate}
  \item Learning the underlying game (a.k.a the complete preference profile)
  \item Finding a stable partition
\end{enumerate}

The rationale is that the closer the learned preference profile is to reality,
the more likely we are to observe theoretically stable partitions matching
the ground truth partition in real-world data, such as party affiliation in
our dataset.
The problem with this approach lies in verifying the fidelity of the learned
preference profile --- we observe that even simple preference profiles such as top
and bottom responsive games have an exponentially large representation in the number of players \footnote{The underlying reason is actually that they have a high VC dimension as noted in \cite{ijcai2017-380}.}
(\autoref{subsec:top_responsive_preferences} and
\autoref{subsec:bottom_responsive_preferences}).
Given our dataset, it means each player's complete preference relation contains
an order of $2^{147} \approx 10^{44}$ coalitions.
Even if we have access to every parliament member, it is unrealistic to ask them
to rank this many possible coalitions.

PAC stability inspires an alternative approach: instead of first learning the complete preference profile then computing a stable partition, we directly
infer a PAC stable partition from the partial preference relations observed in the Knesset data.
The resulting partition does not have as strong stability guarantees as that
of the previous approach: the trade-off we make is to remove the requirement for
the complete preference profile, which is unverifiable given the size of
real-world data, such as our parliament voting data.

Most hedonic game models we consider require players to submit a ranking of
coalitions they belong to.
In the Knesset data however, we observe only approval/disapproval of bills,
therefore we need to formulate ways to translate the voting data to preference
relations of each parliament member, satisfying the requirements of the given
model.
If the formulation results in a complete preference profile, we use it to
compute a deterministic stable partition.
We then attempt to ``PAC-ify'' the formulation to discover a PAC stable partition
directly from sample data, i.e. partial preference relations inferred from
voting behaviors observed in a subset of all bills.

For each PAC model, we simulate i.i.d. by sampling with replacement
$\frac{3}{4}$ of all bills; we repeat the run 50 times to evaluate the
consistency between different runs.
How do we know whether the resulting outputs are robust?
We use information theoretic measures to measure the distance among the 50
partitions.
If most partitions are in agreement, the average information distance is
expected to be small.
We still need to select one or aggregate these 50 partitions into one single
partition as a representative output from each PAC model, which we can use
to compare with the ground truth partition and other models.
We select the conceptual ``centroid'' of the 50 partitions by calculating
the sum of information distance between every partition to other 49 partitions
and selecting the partition with the smallest sum of distance as the
representative.

With one partition produced by every model, we then compare it against our
ground truth, which is party affiliation, both quantitatively and qualitatively.
The quantitative measures give us a quick way to compare across multiple models;
that said, there are qualitative subtleties not captured by the quantitative
measures, such as a parliament member who is known to hold opinions that
mismatch her party ideologies and vote differently, which we will examine
separately.

\subsection{Information Theoretic Measures}
\label{subsec:information_theoretic_measures}

We need to quantify the distance between two given partitions, in order
to select a representative out of multiple runs of a PAC model and to compare
model partitions to the ground truth partition.
We turn to information theoretic measures, which are frequently used in cluster
analysis.
Different from other classes of measures for comparing clusterings, such as
pair-counting based and set-matching based measures, information theoretic
measures have strong mathematical foundation and the ability to detect
non-linear similarities \cite{Vinh:2010:ITM:1756006.1953024}.
The information theoretic measure we are interested in is Variation of
Information proposed by \citenameyear{MEILA2007873}.

Given a partition made of $J$ coalitions $\pi = \{S_1, \cdots S_J\}$, the
probability of a randomly picked player being in coalition $S_j$ is
$\frac{|S_j|}{|N|}$.
The uncertainty (a.k.a. \textit{Entropy}) associated with the given partition
$\pi$ is defined as
$H(\pi) = - \sum^J_{j=1} \frac{|S_j|}{|N|} \log{\frac{|S_j|}{|N|}}$.
It can be interpreted as the average amount of data, in bits for example,
required to encode coalition labels of every player in partition $\pi$ in order
to transmit this partition over a communication channel.

Suppose now we have the structure of another partition
$\pi' = \{S'_1, \cdots, S'_K\}$ made of $K$ coalitions, we can then ask what
would be the amount of data required to encode $\pi$ given $\pi'$
--- this is \textit{Conditional Entropy}:
$H(\pi|\pi') = - \sum^J_{j=1} \sum^K_{k=1} \frac{|S_j \cap S'_k|}{|N|} \log{\frac{|S_j \cap S'_k|/|N|}{|S_k|/|N|}}$.

\begin{description}
    \item[Mutual Information (MI)] represents how much the knowledge of
$\pi'$ helps reduce the number of bits required to encode $\pi$:
$\MI(\pi, \pi') = H(\pi) - H(\pi|\pi')$.
    \item[Variation of Information (VI)] can be interpreted as the total
amount of information change going from partition $\pi$ to partition $\pi'$;
$H(\pi|\pi')$ represent the amount of information we lose about $\pi$
$H(\pi'|\pi)$ represent the amount of information we gain about $\pi'$:
$\VI(\pi, \pi') = H(\pi|\pi') + H(\pi'|\pi)$.
\end{description}

Variation of information can also be expressed using mutual information:
$\VI(\pi, \pi') = H(\pi) + H(\pi') - 2\MI(\pi, \pi')$.
Different from mutual information, variation of information satisfies the
properties of a true metric: non-negativity, symmetry and triangle inequality.
Being a metric makes it a measure that matches our intuitive understanding of
distance.

Variation of information between two partitions has a range of $[0, \log(|N|)]$.
Since VI is a distance measure, the smaller the value the closer two partitions
are; two identical partitions has a VI of 0, while the VI value for two very
different partitions is bounded by logarithm of total number of players to be
partitioned.

Consider three conceptually ``bad'' partitions of the Knesset:

\begin{enumerate}
  \item Singletons: all singleton coalitions
  \item All-in-one: The grand coalition
  \item Random 10: 10 roughly equal sized coalitions with every player randomly
    assigned to one of the coalitions.
    10 corresponds to the number of parties in the Knesset.
\end{enumerate}

Using party affiliations as the ground truth partition, the VIs between the
bad partitions and our ground truth partition are visualize in
\autoref{Experiment:fig:vi_baselines}.

\begin{figure}[ht]
  \centering
  \includegraphics[width=\linewidth]{\Pic{png}{vi_baselines}}
  \caption{The Knesset partition baseline variation of information (VI) values}
  \label{Experiment:fig:vi_baselines}
\end{figure}

Notice under VI, Singletons and All-in-one partitions are not as bad, this is
because despite metrical, VI is not a measure corrected for chance.
For this reason, we adopt another measure named \textit{Adjusted Mutual
Information} (AMI):

\[
  \AMI(\pi, \pi') = \frac{I(\pi, \pi') - E(I(\pi, \pi'))}{\max(H(\pi), H(\pi')) - E(I(\pi, \pi'))}
\]

$E(I(\pi, \pi'))$ is the expected value of mutual information between
the two given partitions $\pi$ and $\pi'$.
Two random partitions have an expected mutual information of 0, their AMI
is therefore around 0.
When two partitions are identical their AMI is 1.
AMI is more sensitive in detecting bad partitions, however it is not metrical
\cite{Vinh:2010:ITM:1756006.1953024}.

We compute the AMI values for our baseline bad partitions on the Knesset.
The resulting values are indeed tightly around zero
(\autoref{experiment:table:ami_baselines}).

\begin{table}[ht]
\centering
\begin{tabular}{|c|c|}
\hline
       & Ajusted Mutual Information \\ \hline
Singletons & 3e-14 \\
All-in-One & -5e-16 \\
Randome 10 & 0.007 \\
\hline
\end{tabular}
\caption{The Knesset partition baseline AMIs}
\label{experiment:table:ami_baselines}
\end{table}

We use both VI and AMI as the quantitative measurements in our experiments,
as VI has the advantage of being a metric and AMI is good for detecting bad
partitions.
