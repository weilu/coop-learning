\SetPicSubDir{ch-Experiment}
\SetExpSubDir{ch-Experiment}

\chapter{Experiment Methodology}
\label{ch:experiment}
\vspace{2em}

\section{Data}
\label{sec:data}

The Israeli political system, in thanks partially to its use of
proportional representation, is made of multiple parties.
The Israeli Knesset is the national legislative branch of the Israeli government.
We limit our study to the 20th Knesset (2015-2019), which is the last
Knesset preceding the current one.
In this Knesset there are 10 parties, but some parties are unions of smaller parties.
In some cases the parties have merged (e.g., the Likud or Meretz),
and in some cases, they have different organizational structure
(e.g., the Jewish Home party or the Joint List).

While in the past, this preponderance of parties led to a multi-dimensional
party system, leading to unexpected coalitions and combinations,
in the past few decades Israeli parties can be, in general,
ordered along a single right-left axis, which has to do mainly with
the parties' approach to the Israeli-Palestinian conflict.
This simplifies the considerations we need to take into account when analyzing
the outcomes, and allows for easier comparison between the different models.
Moreover, it allows us to more easily discern
unexpected coalition structures that need to be explained.
There is also relative ideological cohesion between coalition parties
as well as between opposition parties.

No Israeli party has ever been elected with a majority of Knesset seats,
which means all governments are made of coalitions.
In the past few years (including the Knesset we investigate)
governments have tightened their grip over coalition parties
by having a government committee issue a generally mandatory
voting instructions for every proposed bill.
The opposition, of course, is under no such control,
but due to their relative ideologies, there is significant agreement.

The Knesset website provides data access through Open Data Protocol (OData)
on all its parliament members, laws, and every member's votes on every law.
The Kenesset has 120 seats, but the 20th Knesset has 147 parliament members
due to some Knesset members resigning or joining mid-term.
We download all 147 parliament members' information including name,
party affiliation and their votes for all 7515 bills deliberated.
A vote can take on one of the following values
0 (vote canceled), 1 (vote for), 2 (vote against),
3 (abstained), 4 (did not attend).


\subsection{Data Preprocessing}

During data processing, we notice that more than half of all the vote values
are missing despite the presence of value 4 for ``did not attend''.
It raises questions regarding the data quality.
We discover another API endpoint which provides summary information for
every bill, including total number of ``for'' votes, ``against'' votes,
``abstained'' votes, and if the bill is accepted.
We download bill summary data set for the 20th Knesset and use it
to check against the tallied vote numbers from the individual vote data.
We end up discovering more inconsistencies:

\begin{itemize}
  \item Total bills checked: 7513,
  \item Missing bills in the summary dataset: 2,
  \item Number of bills with pass/reject status inconsistent: 26,
  \item Number of bills with for count inconsistent : 273,
  \item Number of bills with against count inconsistent : 402,
  \item Number of bills with abstain count inconsistent : 2
\end{itemize}

We reach out to Knesset regarding the data inconsistency between
two API endpoints, as well as the missing value issue.
The Knesset data management team state that the inconsistency
is likely a result of manual vote entries instead of votes recorded
as a result of pressing the electronic button.
Such manual vote entries are captured by the bill summary dataset
but not individual member's vote dataset.

For the purpose of our research, we use the dataset with individual member's
votes on every bill since it has the level of granularity required for modeling.
We remove the 26 bills whose pass/reject status is inconsistent between
the summary and individual data sets, since it's most likely that
the individual data is wrong.

% Maybe include data summary tables?
% total bills, total parliament members
% total for, against, abstained, did not attend, missing votes
% per bill: min, max, average for, against, abstained, did not attend, missing votes
% per member: min, max, average for, against, abstained, did not attend, missing votes

\section{Experiment Design}

Recall our first research questions: Can we use hedonic games to model
real-world collaborative activities?
We see in past works, to demonstrate the power of game theoretic models
in describing real-world data, it usually involves the following two steps:

\begin{enumerate}
  \item Learning the underlying game (a.k.a the complete preference profile)
  \item Finding a stable partition
\end{enumerate}

The rationale is that the closer the learned preference profile is to reality,
the more likely we are to observe theoretically stable partition matching
the ground truth parition in real-world data, such as party affiliation in
our dataset.
The problem with this approach lies in verifying the fidelity of the learned
preference profile --- we observe even simple preference profiles such as top
and bottom responsive games are exponentially large in the number of players
(Chapter~\ref{ch:preliminaries} Section~\ref{sec:hedonic_game}
Subsection~\ref{subsec:top_responsive_preferences}
and Subsection~\ref{subsec:bottom_responsive_preferences}).
Given our dataset, it means each player's complete preference relation contains
an order of $2^147 \approx 10^44$ coalitions.
Even if we have access to every parliament member, it is unrealistic to ask them
to rank these many possible coalitions.

PAC stability inspires an alternative approach; instead of first learning the
complete preference profile then computing a stable parition, we directly
infer a PAC stable partition from partial preference relations we observe
each player demonstrates in a dataset.
The resulting partition does not have as strong stability guarantees as that
of the previous approach; the tradeoff we make is to remove the requirement for
the complete preference profile, which is unverifiable given the size of
real-world data, such as our parliament voting data.

For every hedonic game model we investigate, we formulate ways to translate the
voting data to preference relations of each parliament member, satisfying the
requirements of the given model.
If the formulation results in a complete preference profile, we use it to
compute a deterministic stable partition.
We then attempt to PAC-ify the formulation to discover a PAC stable partition
directly from sample data, i.e. partial preference relations inferred from
voting behaviors observed in a subset of all bills.

For each PAC model, we simulate i.i.d. by sampling with replacement $3/4$ of
all bills; repeat the run 50 times to evaluate the consistency between
different runs and for producing a robust PAC stable partition minimizing the
chance of luck.
We use information theoretic measures to measure the distance among the 50
partitions.
If most partitions are in agreement, the average information distance is
expected to be small.
We still need to select one or aggregate these 50 partitions into one single
partition as a representative output from each PAC model, which we can use
to compare with the ground truth partition and other models.
We select the conceptual ``centroid'' of the 50 partitions by calculating
the sum of information distance between every partition to other 49 partitions
and selecting the partition with the smallest sum of distance as the
representative.

With one partition produced by every model, we then compare it against our
ground truth, which is party affiliation, both quantitatively and qualitatively.
The quantitative measures give us a quick way to compare across multiple models;
while there are qualitative subtleties not captured by the quantitative
measures, such as a parliament member who is known to hold opinions that
mismatch her party ideologies and vote differently, which we will examine
separately.

\subsection{Information Theoretic Measures}
We need to be able to quantify distance between two given partitions, in order
to select a representative out of multiple runs of a PAC model and to compare
model partitions to the ground truth partition.
We turn to information theoretic measures, which are frequently used in cluster
analysis.
Different from other classes of measures for comparing clusterings, such as
pair-counting based and set-matching based measures, information theoretic
measures have strong mathematical foundation and the ability to detect
non-linear similarities \cite{Vinh:2010:ITM:1756006.1953024}.
The information theoretic measure we are interested in is Variation of
Information proposed by \citenameyear{MEILA2007873}.

Given a partition made of $x$ coalitions $\pi = \{S_1, \cdots S_x\}$, the
probability of a randomly picked player being in coalition $S_i$ is
$\frac{|S_i|}{|N|}$.
The uncertainty (a.k.a. \textit{Entropy}) associated with the given partition
$\pi$ is defined as
$H(\pi) = - \sum^x_{i=1} \frac{|S_i|}{|N|} \log{\frac{|S_i|}{|N|}}$.
It can be interpreted as the average amount of data, in bits for example,
required to encode coalition labels of every player in parition $\pi$ in order
to transmit this partition over a communication channel.

Given another partition made of $y$ coalitions $\pi' = \{S'_1, \cdots, S'_y\}$,
if we already know $\pi'$, we can then ask what would be the amount of data
required to encode $\pi$ given $\pi'$ --- this is \textit{Conditional Entropy}:
$H(\pi|\pi') = - \sum^x_{i=1} \sum^y_{j=1} \frac{|S_i \cap S'_j|}{|N|} \log{\frac{|S_i \cap S'_j|/|N|}{|S_j|/|N|}}$.

\textit{Mutual Information} represents how much the knowledge of $\pi'$ help
reduce the number of bits required to encode $\pi$:
$\MI(\pi, \pi') = H(\pi) - H(\pi|\pi')$

\textit{Variation of Information} can be interpreted as the total amount of
information change going from partition $\pi$ to partition $\pi'$;
$H(\pi|\pi')$ represent the amount of information we lose about $\pi$
$H(\pi'|\pi)$ represent the amount of information we gain about $\pi'$:
$\VI(\pi, \pi') = H(\pi|\pi') + H(\pi'|\pi)$.
It can also be expressed using mutual information:
$\VI(\pi, \pi') = H(\pi) + H(\pi') - 2\MI(\pi, \pi')$.
Different from mutual information, variation of information satisfies the
properties of a true metric: non-negativity, symmetry and triangle inequality.
Being a metric makes it a measure that matches our intuitive understanding of
distance.
