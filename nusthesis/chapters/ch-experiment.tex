\SetPicSubDir{ch-Experiment}
\SetExpSubDir{ch-Experiment}

\chapter{Experiment Methodology}
\label{ch:experiment}
\vspace{2em}

\section{Data}
\label{sec:data}

The Israeli political system, in thanks partially to its use of
proportional representation, is made of multiple parties.
The Israeli Knesset is the national legislative branch of the Israeli government.
We limit our study to the 20th Knesset (2015-2019), which is the last
Knesset preceding the current one.
In this Knesset there are 10 parties, but some parties are unions of smaller parties.
In some cases the parties have merged (e.g., the Likud or Meretz),
and in some cases, they have different organizational structure
(e.g., the Jewish Home party or the Joint List).

While in the past, this preponderance of parties led to a multi-dimensional
party system, leading to unexpected coalitions and combinations,
in the past few decades Israeli parties can be, in general,
ordered along a single right-left axis, which has to do mainly with
the parties' approach to the Israeli-Palestinian conflict.
This simplifies the considerations we need to take into account when analyzing
the outcomes, and allows for easier comparison between the different models.
More over, it allows us to more easily discern
unexpected coalition structures that need to be explained.
There is also relative ideological cohesion between coalition parties
as well as between opposition parties.

No Israeli party has ever been elected with a majority of Knesset seats,
which means all governments are made of coalitions.
In the past few years (including the Knesset we investigate)
governments have tightened their grip over coalition parties
by having a government committee issue a generally mandatory
voting instructions for every proposed bill.
The opposition, of course, is under no such control,
but due to their relative ideologies, there is significant agreement.

The Knesset website provides data access through Open Data Protocol (OData)
on all its parliament members, laws, and every member's votes on every law.
The Kenesset has 120 seats, but the 20th Knesset has 147 parliament members
due to some Knesset members resigning or joining mid-term.
We download all 147 parliament members' information including name,
party affiliation and their votes for all 7515 bills deliberated.
A vote can take on one of the following values
0 (vote canceled), 1 (vote for), 2 (vote against),
3 (abstained), 4 (did not attend).


\subsection{Data Preprocessing}

During data processing, we notice that more than half of all the vote values
are missing despite the presence of value 4 for ``did not attend''.
It raises questions regarding the data quality.
We discover another API endpoint which provides summary information for
every bill, including total number of ``for'' votes, ``against'' votes,
``abstained'' votes, and if the bill is accepted.
We download bill summary data set for the 20th Knesset and use it
to check against the tallied vote numbers from the individual vote data.
We end up discovering more inconsistencies:

\begin{itemize}
  \item Total bills checked: 7513,
  \item Missing bills in the summary dataset: 2,
  \item Number of bills with pass/reject status inconsistent: 26,
  \item Number of bills with for count inconsistent : 273,
  \item Number of bills with against count inconsistent : 402,
  \item Number of bills with abstain count inconsistent : 2
\end{itemize}

We reach out to Knesset regarding the data inconsistency between
two API endpoints, as well as the missing value issue.
The Knesset data management team state that the inconsistency
is likely a result of manual vote entries instead of votes recorded
as a result of pressing the electronic button.
Such manual vote entries are captured by the bill summary dataset
but not individual member's vote dataset.

For the purpose of our research, we treat the data with individual member's
votes on every bill as the ground truth since it has the level of granularity
required for modeling.
We remove the 26 bills whose pass/reject status is inconsistent between
the summary and individual data sets, since it's most likely that
the individual data is wrong.

\section{Quantitative Measurements}
\section{Qualitative Assessment Criteria}
