\SetPicSubDir{ch-Intro}

\chapter{Introduction}
\label{ch:intro}

Consider a parliament; who are ``natural allies'' amongst the parliament members?
Assuming each member has preferences over each other and possible groups they may
form, can we discover political alliances such that people are content with the 
group they belong to, resulting in a stable situation where nobody has any
incentive to seek alternative alliance?
What if we only observe partial data about each parliament member's preferences;
are we still able to derive a set of stable clusters with confidence? 

Such a task of partitioning people into groups is commonly known as cluster
analysis, which is an essential task of exploratory data mining.
It has been studied extensively by the machine learning community.
However, most works that took the data driven approach completely ignore an
individual player's utility and their strategic behaviors.
Meanwhile in the field of game theory, the computational social choice
community puts in considerable efforts studying the formation of stable
partitions under various utility models for individual players (known as
coalition formation games); however most works are purely theoretical.
In addition, many existing stable solution finding algorithms require full
visibility of all agents' preferences, which is a major obstacle in applying
the research on real-world collaborative activities.

\citenameyear{ijcai2017-380} lay the theoretical foundation of Probably
Approximately Correct (PAC) stability; they provide an algorithm which
produces a partition that is likely resistant to deviation given limited
knowledge of agents' hedonic preferences as input.  
This opens the possibility of constructing models using a data driven
approach, while accounting for agents' strategic behaviors.

We present the first application of strategic hedonic games models on
real-world data.
We address the following research questions:
\begin{enumerate}
    \item Can we use hedonic games to model real-world collaborative
      activities? (Chapters~\ref{ch:experiment} and~\ref{ch:hedonic})
    \item How well does the outcome compare to ground truth?
      (Sections~\ref{sec:quantitative_analysis}
      and~\ref{sec:qualitative_analysis})
    \item How well does the outcome compare to that of canonical clustering
      and community detection models? (Chapter~\ref{ch:comparison})
\end{enumerate}

\section{Thesis Synopsis}
Our contribution includes an algorithmic improvement for finding a PAC stable
solution for top responsive games, a sub-class of hedonic games
(Section~\ref{subsec:improved_pac_algorithm}).
We take the Israeli parliament (the Knesset) members' past voting records on
legislative bills (Section~\ref{sec:data}), design experiments
where we construct various hedonic game
models (Section~\ref{sec:experiment_design}), and apply the improved algorithm, when applicable, to produce PAC stable
partitions (Chapter~\ref{ch:hedonic}).
We detail in our analysis (Chapter~\ref{ch:analysis}), by using the party
affiliations as the ground truth partition, our PAC partitions are able to
reveal the Knesset members' political positions at large.
Comparing to $k$-means clustering and stochastic block models
(Chapter~\ref{ch:comparison}), three of our PAC models reveals politicians
whose voting behaviors are known to break from their party affiliations and
whose allegiance is fickle.

The rest of this thesis is organized as follows:
In \autoref{ch:review}, we present a summary of both literature our study is
built upon, as well as other interesting theoretical and empirical works in the area.
\autoref{ch:preliminaries} provides a systematic account of relevant formal
definitions and examples on the topic of hedonic games and stability solution
concepts.
Lastly, we conclude the entire thesis and discuss directions for future
research in \autoref{ch:concl}.
