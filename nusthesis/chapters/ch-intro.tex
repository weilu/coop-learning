\SetPicSubDir{ch-Intro}

\chapter{Introduction}
\vspace{2em}

Cluster analysis is an essential task of exploratory data mining.
It has been studied extensively by the machine learning community.
However, most works that took the data driven approach completely ignore an individual player's utility and their strategic behaviors.
Meanwhile in the field of game theory, the computational social choice community puts in considerable efforts studying stability related solution concepts of coalition formation games with hedonic preferences;
however most works are purely theoretical.
In addition, existing stable solution finding algorithms require full visibility of all agents' preferences, which is a major obstacle in applying the research on real-world collaborative activities.

Recent research by \citenameyear{ijcai2017-380} lays down the theoretical foundation of Probably Approximately Correct (PAC) stability;
they provide an algorithm which produces a partition that is likely resistant to deviation given limited knowledge of agents' hedonic preferences as input.
This opens the possibility of constructing models taking the data driven approach, and at the same time accounts for agents' strategic behaviors.

We present the first application of strategic hedonic games models on real-world data.
We answer the following research questions: Can we use hedonic games to model real-world collaborative activities?
How well does the outcome compare to ground truth?
How well does the outcome compare to that of canonical clustering and community detection models?


\section{Thesis Synopsis}
Our contribution includes an algorithmic improvement for finding a PAC stable solution for top responsive games, a sub-class of hedonic games
% (Section~\ref{sec:improved_algo})
.  We take the Israeli parliament (the Knesset) members' past voting records on legislative bills 
% (Section~\ref{subsec:data}),
construct top responsive preference models, and apply the improved algorithm to produce PAC stable partitions
% (Section~\ref{subsec:exp_pac_models})
.  We detail in our analysis 
% (Section~\ref{subsec:exp_analysis})
, by using the party affiliations as the ground truth partition,
our PAC partitions are able to reveal the Knesset members' political positions at large. 
Comparing to $k$-means clustering and stochastic block models 
% (Section~\ref{subsec:exp_comparison_models})
, one of our PAC models reveals politicians whose voting behaviors are known to break from their party affiliations and whose allegiance is fickle.

% The rest of this thesis is organized as follows. 
% In \autoref{ch:review}, we conduct a literature review. 
% \autoref{ch:rice} provides the study on rice. 
% \autoref{ch:noodle} describes noodles. 
% We conclude the entire thesis as well as discuss further directions for future research in \autoref{ch:concl}.
