\chapter{Literature Review}
\label{ch:review}

Past works on hedonic preferences focus on the existence and computational
aspect of stability concepts in various sub-classes of hedonic coalition
formation games
\cite{Aziz:2012:ESH:2343776.2343806}, \cite{aziz_savani_moulin_2016},
\cite{Aziz:2016:BHG:3032027.3032048}.

Specifically on the topic of top responsive games,
\citenameyear{ALCALDE2004869} show that top responsive games guarantee the
existence of a core stable partition, which is discoverable through the Top
Covering Algorithm.
\citenameyear{DIMITROV2007130} simplify the original Top Covering Algorithm
and proved that it not only constructs a core, but a strict core.
\citenameyear{Dimitrov2006TopRA} also show that adding the mutuality condition
to the simplified top covering algorithm produces a Nash stable partition.
\citenameyear{Aziz:2012:ESH:2343776.2343806} further prove that with mutuality the partition produced
is in fact strict strong Nash stable for any top responsive game.

%TODO: Boolean & Bottom Responsive

The abovementioned works build solid theoretical foundations for various
classes of hedonic games, however all of them assumes knowledge of every
player's preferences, which is often unattainable in real-world scenario.

Some AI/economics researchers focus their attention on allowing uncertainty in
player valuation of coalitions in hedonic games.
\citenameyear{Chalkiadakis2004, Chalkiadakis:2008:SDM:1402383.1402435} take the
Bayesian approach where every agent derives value for coalitions through
reasoning about the capabilities of other agents and possible actions that a
coalition may take.
Their reinforcement learning model is based on the assumption that players
learn about other players' preferences through repeated interactions; the model
produces a solution that converges to a \textit{Bayesian core}.
\citenameyear{Chalkiadakis2007} subsequently extend Bayesian core to
non-cooperative games through coalitional bargaining.
Extending the concept of PAC (Probably Approximately Correct) learning,
\citenameyear{Balcan:2015:LCG:2832249.2832315} introduce the concept of PAC
core stability.
Building on top of the PAC stability solution concept, recent work by
\citenameyear{ijcai2017-380} establishes the existance of PAC core solutions
of top responsive games and proposes an algorithm to find such PAC solutions.

% TODO: application of cooperative game theory model on real world data

Our empirical work in this thesis is based upon the theoretical works on PAC
stability.
This thesis, to a minor degree, improves the PAC core finding algorithm in
\cite{ijcai2017-380} (Section~\ref{subsec:improved_pac_algorithm}) on top
responsive games and extends it to apply in bottom responsive games as well as
Boolean hedonic games.
More importantly, we demonstrate, using real-world parliament data, the
efficacy of PAC hedonic game theoretical models (See Chapter~\ref{ch:analysis}).
