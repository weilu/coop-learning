\chapter{Literature Review}
\label{ch:review}

Past works on hedonic preferences focus on the existence and computational
aspect of stability concepts in various sub-classes of hedonic coalition
formation games
\cite{Aziz:2012:ESH:2343776.2343806}, \cite{aziz_savani_moulin_2016},
\cite{Aziz:2016:BHG:3032027.3032048}.
The three sub-classes we are especially interested in are top responsive games,
bottom responsive games, and Boolean hedonic games.

Specifically on the topic of top responsive games,
\citenameyear{ALCALDE2004869} show that top responsive games guarantee the
existence of a core stable partition, which is discoverable through the Top
Covering Algorithm.
\citenameyear{DIMITROV2007130} simplify the original Top Covering Algorithm
(Algorithm~\ref{alg:top_covering}) and proved that it not only constructs a
core, but a strict core.
\citenameyear{Dimitrov2006TopRA} also show that adding the mutuality condition
to the simplified top covering algorithm produces a Nash stable partition.
\citenameyear{Aziz:2012:ESH:2343776.2343806} further prove that with mutuality
the partition produced is in fact strict strong Nash stable for any top
responsive game.
We discuss these works in detail in
Section~\ref{subsec:top_responsive_preferences}.

\citenameyear{SuSu10} first introduce the concept of \textit{bottom refuseness},
(a.k.a.\ bottom responsiveness) as a counterpart of top responsiveness, for
modeling conservative players.
They also show that such a game has a core stable solution and is discoverable
through a procedure named ``bottom avoiding algorithm''
(Algorithm~\ref{alg:bottom_responsive_core}).
For a more in-depth recap of bottom responsive games, please refer to
Section~\ref{subsec:bottom_responsive_preferences}.

\citenameyear{Aziz:2016:BHG:3032027.3032048} propose Boolean hedonic games,
which is characterized by a very simple preference model.
They prove that Boolean games guarantee the existence of a core stable solution;
they include the core finding procedure (Algorithm~\ref{alg:boolean_core}) as
part of their proof.
We present relevant concepts and examples in
Section~\ref{subsec:boolean_preferences}.

The abovementioned works build solid theoretical foundations for various
classes of hedonic games, however all of them assumes knowledge of every
player's preferences, which is often unattainable in real-world scenario.
Some AI/economics researchers focus their attention on allowing uncertainty in
player' valuations of coalitions in hedonic games.

Chalkiadakis and Boutilier take the Bayesian approach where every agent derives
value for coalitions through reasoning about the capabilities of other agents
and possible actions that a coalition may take
\cite{Chalkiadakis2004, Chalkiadakis:2008:SDM:1402383.1402435}.
Their reinforcement learning model is based on the assumption that players
learn about other players' preferences through repeated interactions; the model
produces a solution that converges to a \textit{Bayesian core}.
\citenameyear{Chalkiadakis2007} subsequently expand Bayesian core to
non-cooperative games through coalitional bargaining.

Taking a different direction, \citenameyear{Balcan:2015:LCG:2832249.2832315}
extend the concept of PAC (Probably Approximately Correct) learning and
propose the concept of PAC core stability.
Building on top of the PAC stability solution concept, recent work by
\citenameyear{ijcai2017-380} establishes the existance of PAC core solutions
of top responsive games and provides an algorithm to find such PAC solutions.

In this thesis, we base our uncertainty models upon the theoretical works on
PAC stability.
We also improve the PAC core finding algorithm in \cite{ijcai2017-380}
(Section~\ref{subsec:improved_pac_algorithm}) on top responsive games and
extends it to apply to other classes of hedonic games including bottom
responsive games and Boolean hedonic games.
More importantly, we demonstrate, using real-world parliament data, the
efficacy of PAC hedonic game theoretical models (See Chapter~\ref{ch:analysis}).

It is also worth mentioning that all past empirical works on hedonic games we
can find are prescriptive --- hedonic games are used to model agents'
strategic/selfish behaviors; agents are allowed to self-organize, resulting in
stable reource partitions that are also efficient
\cite{5674046, 5137409, 6846512}.
Whereas our empirical study is primarily descriptive --- taking real voting
data of the Israeli parliament, ``translating'' it into hedonic preferences for
each parliament member, searching for a stable partition, and eventually
evaluating the efficacy of the model partition by comparing it to a ground
truth partition.
