\chapter{Literature Review}
\label{ch:review}

\citenameyear{10.2307/1912943} first introduce the concept of hedonic games,
which is a coalition formation game where every player has preference rankings
of possible groups they may belong and cares only about who is in their own group\footnote{They are indifferent about any other groups.} (See 
\autoref{sec:hedonic_game} for examples and formal definitions).
We mention in \autoref{ch:intro} that most works on coalition formation games
are theoretical; specifically, works on hedonic games focus on the existence
and computational aspect of stability concepts under various player utility
assumptions \cite {BOGOMOLNAIA2002201}, \cite{Aziz:2012:ESH:2343776.2343806},
\cite{aziz_savani_moulin_2016}, \cite{Aziz:2016:BHG:3032027.3032048}.
Different player utility models characterize preference models that are
sub-classes of hedonic games.
In particular, this thesis is primarily concerned with three sub-classes,
namely, top responsive, bottom responsive, and Boolean hedonic games.
(We present their definitions and examples in 
\autoref{subsec:top_responsive_preferences}, 
\autoref{subsec:bottom_responsive_preferences}, and 
\autoref{subsec:boolean_preferences} respectively)

On the topic of top responsive games, where a player derives their utility 
from their ``best friends'' (most preferred subset of players) that are grouped with them,
\citenameyear{ALCALDE2004869} show that top responsive games guarantee the
existence of a core stable\footnote{A type of stability solution concept, see
definition in \autoref{subsec:core_strict_core}} partition, which is 
discoverable through the Top Covering Algorithm.
\citenameyear{DIMITROV2007130} simplify the original Top Covering Algorithm
(Algorithm~\ref{alg:top_covering}) and proved that it not only constructs a
core, but a strict core, which is a stronger notion of stability.
\citenameyear{Dimitrov2006TopRA} show that adding the mutuality condition (See
\autoref{sec:stability_concepts}) to the simplified top covering algorithm
produces a Nash stable partition.
\citenameyear{Aziz:2012:ESH:2343776.2343806} further prove that with mutuality
the partition produced is in fact strict strong Nash stable for any top
responsive game.
Both strong Nash and strict strong Nash stability are even stronger stability
notions than core stability; they are discussed in  
\autoref{subsec:strong_nash_strict_strong_nash}.

\citenameyear{SuSu10} first introduce the concept of \textit{bottom refuseness},
(a.k.a.\ bottom responsiveness) as a counterpart of top responsiveness, for
modeling conservative players.
They also show that such a game has a core stable solution and is discoverable
through a procedure named ``bottom avoiding algorithm''
(Algorithm~\ref{alg:bottom_responsive_core}).
For a more in-depth recap of bottom responsive games, please refer to
Section~\ref{subsec:bottom_responsive_preferences}.

\citenameyear{Aziz:2016:BHG:3032027.3032048} propose Boolean hedonic games,
which is characterized by a very simple preference model.
They prove that Boolean games guarantee the existence of a core stable solution;
they include the core finding procedure (Algorithm~\ref{alg:boolean_core}) as
part of their proof.
We present relevant concepts and examples in
Section~\ref{subsec:boolean_preferences}.

The above-mentioned works build solid theoretical foundations for various
classes of hedonic games, however all of them assume knowledge of every
player's preferences, which is often unattainable in real-world scenario.
Some AI and economics researchers focus their attention on allowing uncertainty in
player' valuations of coalitions in hedonic games.

Chalkiadakis and Boutilier take the Bayesian approach where every player derives
value for coalitions through reasoning about the capabilities of other players
and possible actions that a coalition may take
\cite{Chalkiadakis2004, Chalkiadakis:2008:SDM:1402383.1402435}.
Their reinforcement learning model is based on the assumption that players
learn about other players' preferences through repeated interactions; the model
produces a solution that converges to a \textit{Bayesian core}.
\citenameyear{Chalkiadakis2007} subsequently expand Bayesian core to
non-cooperative games through coalitional bargaining.

Taking a different direction, \citenameyear{Balcan:2015:LCG:2832249.2832315}
extend the concept of PAC (Probably Approximately Correct) learning and
propose the concept of PAC core stability.
Building on top of the PAC stability solution concept, recent work by
\citenameyear{ijcai2017-380} establishes the existence of PAC core solutions
of top responsive games and provides an algorithm 
(Algorithm~\ref{alg:pac_top_covering}) to find such PAC solutions.
We present a gist of their works in \autoref{sec:pac_learning_pac_stability}.
Our experiments in this thesis base the uncertainty models upon these theoretical
works on PAC stability; we also extend the PAC top covering algorithm beyond top responsive games to apply on bottom
responsive games (\autoref{sec:bottom_responsive_game}) and Boolean hedonic games
 (\autoref{sec:boolean_hedonic_game}).

It is also worth mentioning that all past empirical works on hedonic games are 
prescriptive --- if agents were to behave according to a utility model, one can
expect to observe a certain outcome.
For example, hedonic games are used to model players' strategic/selfish behaviors;
players are allowed to self-organize, resulting in stable resource partitions that
are also efficient \cite{5674046, 5137409, 6846512}.
Whereas our empirical study is primarily descriptive --- given real-world data,
can we find a suitable model to explain people's behaviors?
Taking past voting data of the Israeli parliament, in our case, we ``translate''
it into hedonic preferences for each parliament member, search for a stable
partition, and eventually evaluate the efficacy of the model partition by
comparing it to party affiliations which we regard as the ground truth partition.