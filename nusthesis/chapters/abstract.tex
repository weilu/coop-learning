\begin{abstract}

Most frameworks for computing solution concepts in hedonic games are theoretical
in nature, and assume complete knowledge of all agent preferences which is 
impractical in real world settings.
Recent work introduces the theoretical foundation of Probably Approximately
Correct (PAC) stability to model stability under uncertainty through sampling.
In this paper, we present the first application of strategic hedonic game models
on real-world data.
We discover that PAC stable solutions under various preference models, including 
top responsive, bottom responsive, and Boolean preferences, reflect Knesset
members' political positions at large, when compared against ground truth party 
affiliations.
Moreover, these comparisons also reveal politicians whose voting behaviors are
known to deviate from party lines.
Finally, we show that PAC hedonic game models compare favorably to both
$k$-means clustering and stochastic block models, which do not account for
strategic behavior, for uncovering voting patterns.

\keywords{
  hedonic games,
  game theory,
  probably approximately correct,
  core stable solution concepts,
  cluster analysis,
  empirical study
}

\end{abstract}