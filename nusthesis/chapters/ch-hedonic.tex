\SetPicSubDir{ch-Hedonic}
\SetExpSubDir{ch-Hedonic}

\chapter{Hedonic Game Stability Models}
\label{ch:hedonic}
\vspace{2em}

\section{Handcrafted Value Function}

First we attempt to formulate a generic top responsive preference model
through representing every player's utility model using a value function.
For each bill there are two natural coalitions:
one formed by parliament members who voted ``for'',
and the other formed by parliament members who voted ``against''.
Note that the values assigned by members to their respective coalitions
are not immediately well defined,
so we define the player $i$'s value function for coalition $S$ as follow:

Let $S_f$ be the set of members who voted ``for'' and
$S_a$ the set of members who voted ``against''.
$S$ can take on one of the two values $\{S_f, S_a\}$ at a time.
$S_p$ denotes the set of parliament members who participated: $S_p = S_f \cup S_a$.

\[
  v_i(S) =
  \begin{dcases}
      1 + \frac{1}{|S|} + \frac{|S_p|}{|N|},& \text{if $S$ is the winning majority}\\
      0,              & \text{otherwise}
  \end{dcases}
\]

$S_f$ is the winning majority if $|S_f| > |S_a|$, vice versa. When $|S_f| = |S_a|$, $S_a$ is considered to be the winning majority.

The conditional value function captures that a winning coalition,
being successfully passing a bill or successfully blocking a bill,
is always worth more than a losing coalition.
$\frac{1}{|S|}$ reflects that a win is considered more valuable when it's
achieved with fewer members, which is also conceptually consistent with
top responsive game's preference for smaller coalitions.
The partition term $\frac{|S_p|}{|N|}$ gives a win more value when more
parliament members voted for or against the bill.
For simplicity, we assume every member in a given coalition assigns
equal value to their coalition.

This handcrafted value function allows us to construct a partial preference
relation for every parliament member.
Note that the handcrafted value function does not guarantee that the generated partial preference profile satisfies top responsiveness.
It makes this approach analogous to improper learning (a.k.a.\ representation independent learning).
In learning theory, improper learning is when a learning algorithm is not
restricted to output a hypothesis from the given hypothesis class,
but should output a hypothesis with error not too far from that of
the best hypothesis from the given hypothesis class.
Similar to the rationale behind improper learning, allowing the
partial preference profile to be not strictly top responsive gives us
the flexibility to choose an approperiate representation of the problem.

Although the original top covering algorithm takes the full preference
profile as input, it would still terminate and produce a partition
if the input is a partial preference profile.
The produced partition is core stable with respect to the provided partial
preference profile.

In our implementation, we assume any omitted coalition is less preferred
than those explicitly stated, because the existence of an observed
coalition is evidence that every members in that coalition prefers to be
a part of that coalition than some unobserved coalition.
Upon discovery of the smallest $\CC$ in each iteration, we also remove any coalition that contains any of the removed player from the remaining player's preference relation. This quickly shrinks the number of preference relations for each remaining member in every iteration of the algorithm.

For the PAC version of the algorithm, we do not have nearly enough samples
as required by the algorithm --- The algorithm requires $O(2n^4)$ samples
which is in the order of $10^5$; we only have 7489 bills in total.
Therefore we simulate i.i.d by sampling $\frac{3}{4}$ of the bills
with replacement and repeated the algorithm run 50 times to check
the consistency of the partitions produced.

\section{Appreciation of Friends}
\section{Aversion to enemies}
\section{Boolean Hedonic Game}

Boolean hedonic preferences means that each player partitions all possible
coalitions they may be part of into two equivalence classes:
satisfactory or unsatisfactory; in other words, a player either likes to be
a member of a coalition or dislikes it.

We can naturally derive such preferences from parliament members' past voting
activities --- whenever a member votes with either ``for'' (or ``against''), they
express their preference to be with other members whose votes are the same
as theirs, and not to be with the group that voted differently.

For example, given 3 players and 3 bills, their votes are as follow:

\begin{table}[h!]
\centering
\begin{tabular}{|c|c|c|c|}
\hline
       & player 1 & player 2 & player 3 \\ \hline
bill 1 & 1 & 2 & 1 \\
bill 2 & 3 & 1 & 1 \\
bill 3 & 1 & 2 & 2 \\
\hline
\end{tabular}
\end{table}

From bill 1, we have $\{1, 3\} \succ_1 \{1, 2\}$, $\{2\} \succ_2 \{1, 2, 3\}$,
$\{1, 3\} \succ_3 \{2, 3\}$.
From bill 2, we have $\{2, 3\} \succ_2 \{2\}$, $\{2, 3\} \succ_3 \{3\}$.
Player 1 is not involved as they voted abstained.
Since there is nobody voted ``2 - against'', both player 2 and player 3 dislike
joining an empty coalition.
From bill 3, we have $\{1\} \succ_1 \{1, 2, 3\}$, $\{2, 3\} \succ_2 \{1, 2\}$,
$\{2, 3\} \succ_3 \{1, 3\}$.

In order to combine the preferences from bill 1, 2 and 3, we need to handle
the cases when a coalition is both liked and disliked by a player, for example,
player 3 likes $\{1, 3\}$ according to bill 1 but dislikes it according to bill 3;
player 3 also likes $\{2, 3\}$ according to bill 2 and 3 but dislikes it according to bill 1.
We tally the number of times a given coalition is liked and disliked by a player and
rule according to the preference with more occurrences. When there are equal number of
likes and dislikes, we break tie in favor of dislike.
As such, in the above example, player 3 dislikes $\{1, 3\}$ and likes $\{2, 3\}$.

Therefore we can merge the preferences derived from bill 1, 2, and 3:

$\{1, 3\} \sim \{1\} \succ_1 \{1, 2\} \sim \{1, 2, 3\}$,
$\{2, 3\} \succ_2 \{2\} \sim \{1, 2\} \sim \{1, 2, 3\}$,
$\{2, 3\} \succ_3 \{1, 3\} \sim \{3\}$

