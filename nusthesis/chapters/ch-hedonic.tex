\SetPicSubDir{ch-Hedonic}
\SetExpSubDir{ch-Hedonic}

\chapter{Hedonic Game Stability Models}
\label{ch:hedonic}

In this chapter we describe the construction of hedonic game models based on
finding stable partitions of the parliament members given the Knesset voting
data.
We explore three preference models each with its own utility assumptions,
namely top responsive games (\autoref{sec:top_responsive_game}), bottom
responsive games (\autoref{sec:bottom_responsive_game}), and Boolean hedonic
games (\autoref{sec:boolean_hedonic_game}).
Under each preference model, we ``translate'' votes accordingly into preferences
for every parliament member; if there are existing algorithms for stable
partition discovery, we implement them and execute them on our dataset;
if there is no existing algorithm, such as the PAC settings for bottom
responsive games and Boolean games, we formulate algorithms ourselves before
implementation and execution.
We also highlight our assumptions, design choices and their rationales for every
model.

\section{Top Responsive Games}
\label{sec:top_responsive_game}
Recall that top responsiveness models a preference profile where every player
derives their utility for a given coalition based on only the most preferred
subset of players in their coalition.
In the context of our dataset, it models the idea that politicians care about
whose votes they stand with; they want to vote with other politicians they like.
Since we do not observe quantifiable preferences of each of the Knesset member
over other members, we derive their preferences from the Knesset voting data
assuming top responsiveness.
Top responsiveness guarantees a strictly core stable partition.
With the addition of mutuality, as discussed in \autoref{sec:stability_concepts},
a strict strong Nash stable partition can be computed if the full preference
profile is given.
When the complete preference profile is not available, we can approximate
a strict strong Nash stable partition.

We will first introduce existing top responsive game algorithms for producing a
stable partition and a PAC stable partition in \autoref{subsec:algorithms},
followed by our improved PAC algorithm in
\autoref{subsec:improved_pac_algorithm}.
We then present two specific top responsive game models that we formulate and
implement as part of our experiments in
\autoref{subsec:handcrafted_value_function} and
\autoref{subsec:appreciation_of_friends}.

\subsection{Existing Algorithms}
\label{subsec:algorithms}
In this section we describe two classes of algorithms which we implement and adapt
for our top responsive preference models.
The full information algorithms require the complete preference profile as
input, whereas the PAC algorithm requires samples of coalitions observed as input.

\paragraph{Full Information Algorithms}
\label{para:full_pref_algos}
The top covering algorithm\cite{ALCALDE2004869, Dimitrov2006TopRA} computes a
strict core stable partition for top responsive games.
The algorithm works by discovering a stable coalition, adding this coalition
to the output partition, removing players of this coalition from the pool
of all players, then repeating this process until there is no more player left
to be assigned.

\begin{algorithm}[htb]
  \caption{Top Covering Algorithm}
  \label{alg:top_covering}
  \textbf{Input:} A hedonic game satisfying top responsiveness.

  \begin{algorithmic}[1]
  \State $R^1 \leftarrow N$; $\pi \leftarrow \emptyset$.

  \For{$k=1$ to $|N|$}
    \State \label{top_cover:select} Select $S^k$
    \State \label{top_cover:remove} $\pi \leftarrow \pi \cup \lbrace S^k \rbrace$ and $R^{k+1} \leftarrow  R^k \setminus S^k$
    \If {$R^{k+1} = \emptyset$}
      \State \Return $\pi$
    \EndIf
  \EndFor

  \State \Return $\pi$
 \end{algorithmic}
\end{algorithm}

In Algorithm~\ref{alg:top_covering}, to further specify Step~\ref{top_cover:select}
we need to define the concept of \textit{connect component}:
Let $C^1(i, S) = \ch(i, S)$ for player $i \in S, S \subseteq N$,
$C^{t + 1}(i, S) = \underset{j \in C^t(i, S)}{\ch(j, S)}$, where $t$ is a positive
integer; the \textit{connected component} of $i$ with respect to $S$ is
$C^{|N|}(i, S)$, denoted as $\CC(i, S)$.
$\CC(i, S)$ can be interpreted as a subgraph with $i$ as the root node in the graph
induced by $S$ as vertices and directed edges $E$,
$(i, j) \in E$ if $j \in \ch(i, S)$ for all $j \in S$.
Step~\ref{top_cover:select} then expands to the following: select $i\in R^k$ such
that $|\CC(i,R^k)| \leq |\CC(j,R^k)|$ for each $j\in R^k$;
and $S^k\leftarrow \CC(i,R^k)$

Note that the input of the top covering algorithm is the entire preference profile
of all players, which is in the order of $O(2^{|N|})$.
Finding the choice set for every player after each round of player removal at
Step~\ref{top_cover:remove} requires scanning through every remaining player's
preference relation, which makes the most expensive part of this algorithm,
so this algorithm is exponential time in the number of players.

\citenameyear{Dimitrov2006} propose a similar algorithm for finding a strict core
stable coalition structure in polynomial time in the number of players
specifically for appreciation-of-friends preference profiles.
Their algorithm effectively replaces Step~\ref{top_cover:select} of
Algorithm~\ref{alg:top_covering} with finding the largest strongly connected
component (SCC)\footnote{A strongly connected component in a directed graph is
a subgraph in which all vertices are reachable from any other vertex.} in the
graph induced by $R^k$.
We refer to their algorithm as \textit{appreciation-of-friends algorithm}.
Due to the definition of appreciation-of-friends preference profile,
Step~\ref{top_cover:remove} no longer requires scanning through every remaining
player's preference relation for removed players, making the partition
equivalently the strongly connected components of the graph induced by $N$.


\paragraph{PAC Algorithm}
\label{para:pac_algo}

When we do not have full information about the preference profile, but we have
sufficient number of observations of coalitions that are i.i.d. samples from
the probability distribution of all possible coalitions and how much each player
values each coalition in this sample set, a PAC stable partition can be computed
using the following algorithm:

\begin{algorithm}[htb]
  \caption{PAC Top Covering Algorithm}
  \label{alg:pac_top_covering}
  \textbf{Input:} $\eps$, $\delta$, set $\cal S$ of $m = (2n^4 + 2n^3)\ceil{\frac{1}{\eps}\log\frac{2n^3}{\delta}}$ samples from $\cal D$
  \begin{algorithmic}[1]

  \State $R^1 \gets N$, $\pi \gets \emptyset$
  \State $\samples \gets \ceil{2n^2 \frac{1}{\eps}\log\frac{2n^3}{\delta}}$
  \For{$k=1$ to $|N|$}

    \State \label{pac_top_cover:sample_begin} $\cal S' \gets$ take and remove $\samples$ samples from $\cal S$
    \State $\cal S' \gets \{T: T \in \cal S', T \subseteq R^k\}$
    \For{$i \in R^k$}
      \If{$i \notin \bigcup_{X \in \cal S'} X$}
        \State$B_{i,k} \gets \{i\}$
      \Else
        \State \label{pac_top_cover:argmax} $B_{i,k} \in \arg\max_{T \in \cal S'}{v_i(T)}$
        \State $B_{i,k} \gets \underset{\{T \in \cal S' : \ch(i,T) = \ch(i,B_{i,k})\}}{\bigcap} T$.
      \EndIf
    \EndFor

    \For{$j = 1,\dots,|R^k|$}
      \State $\cal S'' \gets$ take and remove $\samples$ samples from $\cal S$
      \State $\cal S'' \gets \{T: T \in \cal S'', T \subseteq R^k\}$
      \For{$i \in R^k$}
        \State $B_{i,k} \gets B_{i,k} \cap \underset{T \in \cal S'' : \ch(i,T) = \ch(i,B_{i,k})}{\bigcap} T$.
      \EndFor
    \EndFor \label{pac_top_cover:sample_end}

    \State \label{pac_top_cover:select} Select $S^k$
    \State $\pi \leftarrow  \pi \cup \lbrace S^k \rbrace$; and $R^{k+1} \leftarrow  R^k \setminus S^k$
    \If {$R^{k+1} = \emptyset$}
      \State \Return $\pi$
    \EndIf
  \EndFor

  \State \Return $\pi$
 \end{algorithmic}
\end{algorithm}

Step~\ref{pac_top_cover:select} is identical to that of
Algorithm~\ref{alg:top_covering}'s Step~\ref{top_cover:select}, which expands to:
select $i\in R^k$ such that $|\CC(i,R^k)| \leq |\CC(j,R^k)|$ for each $j\in R^k$;
and $S^k\leftarrow \CC(i,R^k)$

In Algorithm~\ref{alg:pac_top_covering}, every sample $j$ consists of a coalition
$S_j \subseteq N$, and the values assigned to $S_j$ by each of its member
$\vec{v}(S_j) = (v_i(S_j))_{i \in S_j}$.
In practice, exact valuation of each coalition by every player is not necessary;
it is sufficient to have preference rankings of all coalitions in the sample set
for every player, because such rankings allow us to identify the coalition that
maximizes each player's utility (Step~\ref{pac_top_cover:argmax}).


\subsection{Improved PAC Algorithm}
\label{subsec:improved_pac_algorithm}
We can make Algorithm~\ref{alg:pac_top_covering} more efficient
by replacing the step of finding the smallest $\CC$ with finding the largest
Strongly Connected Component ($\SCC$).
We name the resulting algorithm \textit{$\SCC$ based PAC top covering algorithm}.
Our improvement is based on \citenameyear{Dimitrov2006}'s appreciation-of-friends
strict core partition discovery algorithm (See \autoref{para:full_pref_algos}).

We first show that the appreciation-of-friends algorithm generalizes to all top
responsive games; then we demonstrate that making the same modification to the
PAC top covering algorithm (Algorithm~\ref{alg:pac_top_covering}) still produces
a PAC stable partition but with improved running time complexity.

\begin{proposition}
\label{prop:scc_generalizes}
  The appreciation-of-friends algorithm produces a strictly core stable partition given any top responsive preference profile.
\end{proposition}

\begin{proof}
Assuming the resulting partition of the $SCC$ procedure $\pi$ is not strictly core
stable, then there exists at least one weakly blocking coalition $S \subseteq N$
and at least one player $i \in S$ prefers $S$ over $\pi(i)$.
Consider the step when the coalition $\pi(i)$ is generated --- it was the largest
strongly connected component in the directed graph induced by remaining players'
choice sets.
In order to form $S$ that is different from $\pi(i)$, either or both of the
following steps must be carried out:

\begin{enumerate}
  \item at least one player $j \in \pi(i)$ needs to be removed from $\pi(i)$
  \item at least one player $k \notin \pi(i)$ needs to be added to $\pi(i)$
\end{enumerate}

When only Step 1 is carried out, since $\pi(i)$ is a strongly connected component,
$j$ has at least one incoming edge, which means it is in the choice set of at
least one other player $j' \in \pi(i)$, therefore its removal makes player
$j' \in S$ worse off, therefore $S$ cannot be weakly blocking $\pi$.

When only Step 2 is carried out, since player $k$ was not part of the strongly
connected component that induces $\pi(i)$, $k$ only has incoming or outgoing edges
between itself and the $\SCC$ but not both.
If $k$ only has outgoing edges to $\SCC$ it means adding $k$ to $\pi(i)$ will only
make $k$ better off, but other players in $\pi(i)$ worse off because
$S = \pi(i) \cup \{k\}$ is bigger in size $|S| > |\pi(i)|$ while for any player
$i' \in S, i' \neq k$ their choice set remains the same, therefore making $S$ less
preferable than $\pi(i)$ for player $i' \in S, i' \neq k$ per definition of top
responsiveness.
Therefore $S$ is not blocking $\pi$ when $k$ only has outgoing edges to $\SCC$.
In the case $k$ only has incoming edges from $\SCC$, it means adding $k$ to
$\pi(i)$ will only make some player $i \in \pi(i)$ better off as $k$ is in their
choice set.
However $k$ will be made worse off by this move, since none of the player in
$\pi(i)$ is in $k$'s choice set.
As such, $S$ is not blocking $\pi$ when only Step 2 is carried out.

When both Step 1 and Step 2 are carried out to form $S$ from $\pi(i)$, since
Step 1 is bound to make some player $j' \in \pi(i)$ worse off, the only way to
compensate $j'$ is by adding someone from their choice set to $\pi(i)$.
Assuming $k$ is in $j'$'s choice set, then $k$ only has incoming edges from the
$\SCC$, which means $k$ will be worse off by joining $\pi(i)$, so $S$ cannot be
blocking $\pi$.
\end{proof}

\begin{theorem}
  The $\SCC$ based PAC top covering algorithm produces a PAC stable partition
  for any top responsive game.
\end{theorem}

\begin{proof}[Proof Sketch]
\citenameyear{ijcai2017-380} already prove that their PAC algorithm efficiently PAC
stabilizes top responsive games.
The correctness of their PAC algorithm relies on the fact that the coalition
selection and deletion procedure based on finding the smallest $\CC$ produces
a core stable partition when there's full visibility to players' choice sets;
the PAC guarantee is an artifact of the choice set approximation via sampling
(Algorithm~\ref{alg:pac_top_covering} Step~\ref{pac_top_cover:sample_begin} to
Step~\ref{pac_top_cover:sample_end}).
Our modification to their algorithm does not change how players' choice sets
are approximated; we only modify how coalitions are selected.
We show in Proposition~\ref{prop:scc_generalizes} that modifying the coalition
selection procedure from $\CC$ to $\SCC$ still produces a strict core stable
partition for top responsive games, therefore the $\SCC$ based PAC algorithm
still produces a PAC stable partition.
\end{proof}

The largest $\SCC$ procedure provides a running time improvement over the
smallest $\CC$ procedure because finding the smallest $\CC$ requires
$\mathcal{O}(|V|(|V| + |E|))$ time while the largest $\SCC$ can be found using
Tarjan's algorithm in linear time $\mathcal{O}(|V| + |E|)$ \cite{Tarjan72depthfirst}.
This running time improvement is only meaningful for the PAC version of the
top covering algorithm due to the input size of the original top covering
algorithm being exponential in the number of players.
In addition, removing more players in the earlier iterations also reduces
the amount of computation required for the later iterations, making our
experiments run much faster than what it would have been with the original
$\CC$ procedure.
For example, the friends model described in
\autoref{subsec:appreciation_of_friends} runs three times faster using the
improved algorithm\footnote{It takes $\sim30$ minutes with the top covering
algorithm, and $\sim10$ minutes with the improved algorithm to run on a 2018
MacBook Air with 1.6GHz i5 processor and 16GB memory}.


\subsection{Handcrafted Value Function}
\label{subsec:handcrafted_value_function}

First we attempt to formulate a generic top responsive preference model
through representing every player's utility model using a value function.
For each bill there are two natural coalitions:
one formed by parliament members who voted ``for'',
and the other formed by parliament members who voted ``against''.
There are also parliament members who voted ``abstained'' or simply did
not vote on certain bills; in these cases, we exclude them from
consideration for the corresponding bills.
We refer to ``for'' and ``against'' votes ``effective votes'' in this thesis.
For the two coalitions induced by the effective votes of a given bill, the
values assigned by members to their respective coalitions are not immediately
well defined, so we define the player $i$'s value function for coalition $S$
as follow:

Let $S_f$ be the set of members who voted ``for'' and
$S_a$ the set of members who voted ``against''.
$S$ can take on one of the two values $\{S_f, S_a\}$ at a time.
$S_p$ denotes the set of parliament members who participated: $S_p = S_f \cup S_a$.

\begin{equation}
\label{eq:value_function}
  v_i(S) =
  \begin{dcases}
      1 + \frac{1}{|S|} + \frac{|S_p|}{|N|},& \text{if $S$ is the winning majority}\\
      0,              & \text{otherwise}
  \end{dcases}
\end{equation}

$S_f$ is the winning majority if $|S_f| > |S_a|$, vice versa. When $|S_f| = |S_a|$, $S_a$ is considered to be the winning majority.

The conditional value function captures that a winning coalition,
being successfully passing a bill or successfully blocking a bill,
is always worth more than a losing coalition.
$\frac{1}{|S|}$ reflects that a win is considered more valuable when it's
achieved with fewer members, which is also conceptually consistent with
top responsive game's preference for smaller coalitions.
The participation term $\frac{|S_p|}{|N|}$ gives a win more value when there
are more effective votes for a given bill.
This captures the idea that participation correlates to the importance of a
bill in the eyes of the parliament members; the more important a bill, the more
valuable the win, be it passing or blocking the bill successfully.
For example, if two bills are both passed with the same number of ``for''
votes $n_f > 2$, and bill A has no ``against'' vote while bill B has $n_f - 1$
``against'' votes, then the ``for'' coalition in bill B derives higher utility
than the ``for'' coalition in bill A in our formulation, as bill B is more
closely contended.
For simplicity, we assume every member in a given coalition assigns
equal value to their coalition.

This handcrafted value function allows us to construct a partial preference
relation for every parliament member.
Note that the handcrafted value function does not guarantee that the generated partial preference profile satisfies top responsiveness.
It makes this approach analogous to improper learning (a.k.a.\ representation independent learning).
In learning theory, improper learning is when the target function is not from
the class of functions that the learning algorithm can possibly output,
regardless, the learning algorithm should output a hypothesis with error not
too far from that of the best hypothesis from the given hypothesis class.
Similar to the rationale behind improper learning, allowing the
partial preference profile to be not strictly top responsive gives us
the flexibility to choose an appropriate representation of the problem.

Although the original top covering algorithm expects the full preference
profile as input, it would still terminate and produce a partition
if the input is a partial preference profile.
The produced partition ensures that any of the observed coalitions would
have no incentive to deviate.

In our implementation, we assign all unobserved coalition the value of zero;
this is to ensure that the values for observed, winning coalitions are always
greater than any unobserved coalition, because intuitively the existence of
an observed coalition is evidence that every member in that coalition prefers
to be a part of that coalition than some unobserved coalition.
Upon discovery of the largest $\SCC$ in each iteration, we also remove any
coalition that contains any of the removed player from the remaining player's
preference relation.
This quickly shrinks the number of preference relations for each remaining
member in every iteration of the algorithm.

For the PAC version of the algorithm, we do not have nearly enough samples
as required by the algorithm --- The algorithm requires $O(2n^4)$ samples
which is in the order of $10^5$; we only have 7489 bills in total.
Therefore we simulate i.i.d by sampling $\frac{3}{4}$ of the bills
with replacement and repeated the algorithm run 50 times to ensure
the consistency of the partitions produced.

Note that when we make the sample size $100\%$ of all bills and sample
with replacement, the PAC algorithm should produce the same partition
as the original top covering algorithm.
Aside from the test cases we manually construct, we also use this property to parameterize the PAC implementation and verify correctness of
our top covering algorithm implementation.


\subsection{Appreciation of Friends}
\label{subsec:appreciation_of_friends}

Our next set of experiments are based on the appreciation of friends model,
which is a proper subset of top responsive games.
In this preference model, a player classifies other players as either friends
or enemies, and prefers any coalition with more friends and fewer enemies:

More formally, let $G_i$ be player $i$'s set of friends, and $B_i$ the set of enemies. $G_i \cup B_i \cup i = N$ and $G_i \cap B_i = \emptyset$. A preference profile $P^f$ is based on \textit{appreciation of friends} if for all player $i \in N$, $S \succeq_i T$ if and only if $|S \cap G_i| > |T \cap G_i|$ or $|S \cap G_i| = |T \cap G_i|$ and $|S \cap B_i| \leq |T \cap B_i|$.

We define friends of a player's as anyone whose votes agreed with the given
player's more often than they disagreed.
Agreed votes are only counted if the given player voted ``for'' or ``against''.
We experimented with two different ways of counting the disagreed votes:

\begin{enumerate}
  \item Narrow disagreement (general friends): the other player's vote is different from mine, and is either ``for'' or ``against''
  \item Broad disagreement (selective friends): the other player's vote is different from mine
\end{enumerate}

Compared to narrow disagreement, broad disagreement leads to every player being
more selective of friends; it may also break the symmetry of the friend
relationship.
In both cases, for any player, those who are not friends are considered enemies.

\begin{example}
\label{example:votes_friends}
  For example, given 3 players and 3 bills, their votes are as follow:

  \begin{table}[ht]
  \centering
  \begin{tabular}{|c|c|c|c|}
  \hline
         & player 1  & player 2 & player 3 \\ \hline
  bill A & for       & for      & against \\
  bill B & abstained & for      & abstained \\
  bill C & abstained & against  & abstained \\
  \hline
  \end{tabular}
  \end{table}

\end{example}

Under general friends, players 1 and 2 are friends because they agreed on bill A.
In comparison, under selective friends, from player 1's point of view, player 2
is still a friend due to bill A, while player 2 considers player 1 an enemy
because player 1 did not vote with her for bill B and C.

The complete preference profile under general friends is therefore:

player 1: $\{1, 2\} \succ_1 \{1, 2, 3\} \succ_1 \{1\} \succ_1 \{1, 3\} $

player 2: $\{1, 2\} \succ_2 \{1, 2, 3\} \succ_2 \{2\} \succ_2 \{2, 3\}$

player 3: $\{3\} \succ_3 \{1, 3\} \sim \{2, 3\} \succ_3 \{1, 2, 3\}$

The complete preference profile under selective friends is as follow:

player 1: $\{1, 2\} \succ_1 \{1, 2, 3\} \succ_1 \{1\} \succ_1 \{1, 3\} $

player 2: $\{2\} \succ_2 \{1, 2\} \sim \{2, 3\} \succ_2 \{1, 2, 3\} $

player 3: $\{3\} \succ_3 \{1, 3\} \sim \{2, 3\} \succ_3 \{1, 2, 3\}$

As soon as the sets of friends and enemies are derived for each parliament member,
we can expand them into a complete preference profile, and apply the top covering
algorithm to obtain a strict strong Nash stable partition for the general friends
preference profile, and a strict core stable partition for the selective friends
preference profile. The notion of stability is weaker for the selective friends
case due to the lack of symmetry.

Aside from the two partitions yielded from the complete preference profiles,
we implement and run two sets of PAC experiments based on the general friends
and selective friends preference assumptions respectively.
Due to sampling in the PAC version of the experiments, we do not observe all
the bills every time we approximate players' preferences;
therefore, for a given player, the sets of friends and enemies can be different
from the complete preference profile cases, and also different between samples.
Since we take the intersections of the best coalitions for any player across
multiple sampling iterations, intuitively the PAC versions are more conservative
in choosing friends compared to their complete profile preference conterpart.
As a result, we expect to observe smaller and more coalitions in comparison.


\section{Bottom Responsive Games}
\label{sec:bottom_responsive_game}

Bottom responsive games are the counterpart of top responsive games, but are
much less studied class of hedonic games.
In the context of our parliament voting dataset, by taking a bottom responsive
model, we are implicitly assuming that parliament members care more about having
no disagreements with people in the set of members who voted with them.

\subsection{Aversion to Enemies}
\label{subsec:aversion_to_enemies}

Aversion to enemies profile is a proper subclass of bottom responsive games
\cite{SuSu10}.
This preference model is closely related to appreciation of friends model
(\autoref{subsec:appreciation_of_friends}).
Same as the appreciation of friends model, a player classifies other players as
either friends or enemies; different from the friends model where a player
derives utility from the number of friends in a coalition, under the aversion
to enemies model, a player prefers a coalition with fewer enemies and only uses
the number of friends to break ties when two coalitions have the same number
of enemies:

Let $G_i$ be player $i$'s set of friends, and $B_i$ the set of enemies.
$G_i \cup B_i \cup i = N$ and $G_i \cap B_i = \emptyset$.
A preference profile $P^e$ is based on \textit{aversion to enemies} if for every
player $i \in N$, $S \succeq_i T$ if and only if $|S \cap B_i| < |T \cap B_i|$
or $|S \cap B_i| = |T \cap B_i|$ and $|S \cap G_i| \geq |T \cap G_i|$.

Given 3 players and their votes in Example~\ref{example:votes_friends},
under general friends, both player 1 and player 2 view player 3 as enemy, and
player 3 reciprocates this relation.
The expanded preference profile is as follows:

player 1: $\{1, 2\} \succ_1 \{1\} \succ_1 \{1, 2, 3\} \succ_1 \{1, 3\} $

player 2: $\{1, 2\} \succ_2 \{2\} \succ_2 \{1, 2, 3\} \succ_2 \{2, 3\}$

player 3: $\{3\} \succ_3 \{1, 3\} \sim \{2, 3\} \succ_3 \{1, 2, 3\}$

Notice that compared to the appreciation of friends model under selective
friends, both players 1 and 2 now prefer their respective singleton coalition
over the grand coalition.
This illustrates that we can expect to observe smaller coalitions in partitions
generated from the enemy models compared to that from the friend models.

Under selective friends, player 1's only enemy remains as player 3, while
player 2 views both player 1 and 3 as enemies; player 3's enemies remain as
player 1 and 2.
The expanded preference profile is therefore:

player 1: $\{1, 2\} \succ_1 \{1\} \succ_1 \{1, 2, 3\} \succ_1 \{1, 3\} $

player 2: $\{2\} \succ_2 \{1, 2\} \sim \{2, 3\} \succ_2 \{1, 2, 3\} $

player 3: $\{3\} \succ_3 \{1, 3\} \sim \{2, 3\} \succ_3 \{1, 2, 3\}$

We implement the following algorithm proposed by \citenameyear{SuSu10} to
compute a core stable coalition structure for a given complete bottom
responsive preference profile:

\begin{algorithm}[htb]
  \caption{Bottom Responsive Game Core Finding Algorithm}
  \label{alg:bottom_responsive_core}
  \textbf{Input:} A bottom responsive game
  \begin{algorithmic}[1]

  \State $S \leftarrow N$; $\pi \leftarrow \emptyset$.
  \While {$S \neq \emptyset$}
    \State \label{bottom_responsive_core:select_begin} Set
      $\Gamma \leftarrow \{S\}$; $\Phi \leftarrow \emptyset$.
    \While {$\Phi = \emptyset$}
      \State $\Gamma \leftarrow \underset{X \in \Phi}{\bigcup} \underset{i \in X}{\bigcup} \{ X \text{\textbackslash} \{j\} | j \in Y \text{ for some } Y \in \Av(i, X)\}$
      \State $\Phi \leftarrow \lbrace X \in \Gamma | \{i\} \in \Av(i, X) \text{ for each } i \in X \rbrace$
    \EndWhile
    \State \label{bottom_responsive_core:select_end} Select a coalition
      $X \in \Phi$
    \State \label{bottom_responsive_core:reduce} Set
      $\pi \leftarrow \pi \cup \lbrace X \rbrace$ and
      $S \leftarrow  S \setminus X$
  \EndWhile
  \State \Return $\pi$

  \end{algorithmic}
\end{algorithm}

For the PAC setting of the aversion to enemies models,
Algorithm~\ref{alg:pac_top_covering} does not directly apply; we need to adapt
Algorithm~\ref{alg:bottom_responsive_core} to produce a PAC stable partition
instead.
Similar to Algorithm~\ref{alg:pac_top_covering}
(Steps~\ref{pac_top_cover:sample_begin} to~\ref{pac_top_cover:sample_end}),
where we approximate the most preferred coalition for every player through
repeated sampling, we can derive each player's set of friends from each round
of sampling and take the intersection of these friend sets across multiple rounds
of sampling as each player's approximated ``true friends''.
Players that are not ``true friends'' are then enemies.
Given the approximated sets of enemies for each player, we execute
Algorithm~\ref{alg:bottom_responsive_core}
(Steps~\ref{bottom_responsive_core:select_begin} to~\ref{bottom_responsive_core:select_end}) to select a PAC stable coalition,
then add this coalition to the output PAC stable partition, and remove players
in this coalition from the game (Step~\ref{bottom_responsive_core:reduce}).
We repeat the approximation of ``true friends'', selection of PAC stable
coalition, addition of coalition and removal of players with the remaining
players until we exhaust all the players.


\section{Boolean Hedonic Games}
\label{sec:boolean_hedonic_game}

Recall Boolean hedonic preferences means that each player partitions all possible
coalitions they are a part of into two equivalence classes:
satisfactory or unsatisfactory; in other words, a player either likes to be
a member of a coalition or hates it (\autoref{subsec:boolean_preferences}).

\citenameyear{Aziz:2016:BHG:3032027.3032048} show that we can discover a core
stable partition for a given Boolean hedonic game using the following procedure:

\begin{algorithm}[htb]
  \caption{Boolean Hedonic Game Core Finding Algorithm}
  \label{alg:boolean_core}
  \textbf{Input:} A Boolean hedonic game
  \begin{algorithmic}[1]

  \State $N' \leftarrow N$; $\pi \leftarrow \emptyset$.
  \While {$N' \neq \emptyset$}
    \State \label{boolean_core:select} Find $S \subset N'$ where all players
      in $S$ find $S$ satisfactory, and the size of $S$ is the largest if there
      are multiple such coalitions.
    \State $\pi \leftarrow \pi \cup \lbrace S \rbrace$ and
      $N' \leftarrow  N' \setminus S$
  \EndWhile
  \State \Return $\pi$

  \end{algorithmic}
\end{algorithm}

Given our parliament voting data, it is natural to formulate satisfactory
coalitions --- for a given bill, those who voted ``for'' and those who voted
``against'' form two coalitions that are satisfactory among its own members.

\begin{example}
\label{example:votes_boolean}
  Given a parliament with 3 players and 3 bills, their votes are as follow:

  \begin{table}[ht]
  \centering
  \begin{tabular}{|c|c|c|c|}
  \hline
         & player 1  & player 2 & player 3 \\ \hline
  bill A & for       & against  & for \\
  bill B & abstained & for      & for \\
  bill C & for       & against  & against \\
  \hline
  \end{tabular}
  \end{table}
\end{example}

From bill A, we have player 1 and 3 being satisfied with coalition $\{1, 3\}$,
and player 2 satisfied with $\{2\}$.
From bill B, player 2 and 3 find $\{2, 3\}$ satisfactory; player 1's preference
is unknown as their vote is neither ``for'' nor ``against''.
In summary, we have the following satisfactory coalitions for every player:

player 1: $\{1, 3\}, \{1\}$

player 2: $\{2\}, \{2, 3\}$

player 3: $\{1, 3\}, \{2, 3\}$

We can simply assume unobserved coalitions as unsatisfactory.
Since a player is indifferent among satisfactory coalitions in a Boolean hedonic
game, such an assumption guarantees that satisfactory coalitions extracted from
voting behaviors are strictly more preferable than those unobserved coalitions.
With this assumption, we can fill in the missing preference relations to
form a complete Boolean preference profile:

player 1: $\{1, 3\} \sim \{1\} \succ_1 \{1, 2\} \sim \{1, 2, 3\}$

player 2: $\{2\} \sim \{2, 3\} \succ_2 \{1, 2\} \sim \{1, 2, 3\}$

player 3: $\{1, 3\} \sim \{2, 3\} \succ_3 \{3\} \sim \{1, 2, 3\}$

When we feed this preference profile to Algorithm~\ref{alg:boolean_core} as
input, we get $\pi = \{\{1, 3\}, \{2\}\}$ as the core stable partition
\footnote{$\pi' = \{\{2, 3\}, \{1\}\}$ is also a core stable partition, depends
on how we break tie among coalitions that are satisfactory for all their members
and are of the same size.}.

We also observe that the preference profile formulated this way is symmetric ---
if a coalition is satisfactory for one player, it is also satisfactory for all
other players within it.
An implication of this symmetry is that the bill with the broadest
support/disapproval also yields the largest coalition.
This means if there is a single bill with support/disapproval among multiple
parties across the political spectrum, we will observe such multi-party
coalition in the final stable partition.
We attempt to break this symmetry with an alternative, more complicated,
formulation without much success.
The details of that alternative formulation is described in
\autoref{append:alternative_boolean}.

We also would like to highlight two conscious design decisions in our
implementation:

\begin{enumerate}
  \item When a player is matched and removed, we remove any satisfactory
    coalitions that contain the removed player, instead of modifying those
    coalitions.
    This is to maintained integrity of the votes to preference relation
    translation; if we were to modify satisfactory coalitions by simply
    deleting removed players from them, we would be constructing satisfactory
    coalitions that are not actually observed from voting behaviors.
  \item After exhausting satisfactory coalitions in our iterations, we place
    the remaining players in singleton coalitions.
    Since a player is indifferent among unsatisfactory coalitions, singleton
    coalitions is the simplest way to partition the remaining players.
    This is in line with the minimum description length principle, where given
    the same explanatory power, the simplest model is selected.
\end{enumerate}

Formulation of the PAC version of the Boolean hedonic games is tricky, because
Algorithm~\ref{alg:pac_top_covering} does not directly apply --- sampling bills
will produce a large number of satisfactory coalitions for every player.
However, if we break the tie to pick one ``best'' coalition for each player at
Step~\ref{pac_top_cover:argmax}, it violates the definition of Boolean
preferences where a player is indifferent among satisfactory coalitions.

% TODO: add visual maybe a van diagram?
Instead of picking one ``best'' coalition we allow for a set of equally good
coalitions from each set of samples of bills.
Rather than taking the intersection of ``best'' coalitions, we take intersections
of sets of equally good coalitions.
The resulting approximated preferences are thus not one coalition per player,
but a set of coalitions per player.

In Step~\ref{pac_top_cover:select}, we follow the same select step as
Algorithm~\ref{alg:boolean_core} (Step~\ref{boolean_core:select}), which picks
the largest coalition that is satisfactory for every member in it.

\citenameyear{jha2019learning} prove that a hedonic game is efficiently PAC
stablizable if and only if there exists an algorithm that produces a partition
that is consistent with the observed samples\footnote{See Theorem 3.2 in the
original paper.}.
Since our PAC algorithm takes the intersection of every set of satisfactory
coalitions for a player at each round of sampling, the resulting set of
satisfactory coalitions is a subset of each player's ``true'' set of preferred
coalitions.
At the end of each iteration, we find the largest satisfactory coalition
to add to the output partition, which means that all players in this coalition
have no incentive to deviate from this assignment\footnote{Recall that in a
Boolean hedonic game, players are indifferent among their satisfactory
coalitions (\autoref{subsec:boolean_preferences}), therefore any satisfactory
coalition assignment is just as good.}.
As such, our PAC core finding algorithm indeed produces a partition that is
consistent with samples evaluated by the given Boolean hedonic game; therefore
the output partition is PAC core stable.
