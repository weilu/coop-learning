\SetPicSubDir{ch-Hedonic}
\SetExpSubDir{ch-Hedonic}

\chapter{Hedonic Game Stability Based Models}
\label{ch:hedonic}
\vspace{2em}

\section{Handcrafted Value Function}
\section{Appreciation of Friends}
\section{Aversion to enemies}
\section{Boolean Hedonic Game}

Boolean hedonic preferences means that each player partitions all possible
coalitions they may be part of into two equivalence classes:
satisfactory or unsatisfactory; in other words, a player either likes to be
a member of a coalition or dislikes it.

We can naturally derive such preferences from parliament members' past voting
activities --- whenever a member votes with either ``for'' (or ``against''), they
express their preference to be with other members whose votes are the same
as theirs, and not to be with the group that voted differently.

For example, given 3 players and 3 bills, their votes are as follow:

\begin{table}[h!]
\centering
\begin{tabular}{|c|c|c|c|}
\hline
       & player 1 & player 2 & player 3 \\ \hline
bill 1 & 1 & 2 & 1 \\
bill 2 & 3 & 1 & 1 \\
bill 3 & 1 & 2 & 2 \\
\hline
\end{tabular}
\end{table}

From bill 1, we have $\{1, 3\} \succ_1 \{1, 2\}$, $\{2\} \succ_2 \{1, 2, 3\}$,
$\{1, 3\} \succ_3 \{2, 3\}$.
From bill 2, we have $\{2, 3\} \succ_2 \{2\}$, $\{2, 3\} \succ_3 \{3\}$.
Player 1 is not involved as they voted abstained.
Since there is nobody voted ``2 - against'', both player 2 and player 3 dislike
joining an empty coalition.
From bill 3, we have $\{1\} \succ_1 \{1, 2, 3\}$, $\{2, 3\} \succ_2 \{1, 2\}$,
$\{2, 3\} \succ_3 \{1, 3\}$.

In order to combine the preferences from bill 1, 2 and 3, we need to handle
the cases when a coalition is both liked and disliked by a player, for example,
player 3 likes $\{1, 3\}$ according to bill 1 but dislikes it according to bill 3;
player 3 also likes $\{2, 3\}$ according to bill 2 and 3 but dislikes it according to bill 1.
We tally the number of times a given coalition is liked and disliked by a player and
rule according to the preference with more occurrences. When there are equal number of
likes and dislikes, we break tie in favor of dislike.
As such, in the above example, player 3 dislikes $\{1, 3\}$ and likes $\{2, 3\}$.

Therefore we can merge the preferences derived from bill 1, 2, and 3:

$\{1, 3\} \sim \{1\} \succ_1 \{1, 2\} \sim \{1, 2, 3\}$,
$\{2, 3\} \succ_2 \{2\} \sim \{1, 2\} \sim \{1, 2, 3\}$,
$\{2, 3\} \succ_3 \{1, 3\} \sim \{3\}$

% \begin{figure}[!t]
%   \centering
%   \includegraphics[width=.5\linewidth]{\Pic{jpg}{noodle}}
%   \vspace{\BeforeCaptionVSpace}
%   \caption{A bowl of noodles.}
%   \label{Noodle:fig:bowl_of_noodles}
% \end{figure}

