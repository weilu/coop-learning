\ifdefined\DoubleSided
  \documentclass[twoside,openright,a4paper,dvipsnames]{nusthesis}
\else
  \documentclass[a4paper,dvipsnames]{nusthesis}
\fi

\dsp % pseudo double spacing

%%% The abstract of the thesis is titled as "Abstract" by default. 
%%% However, the Graduate Division may require you to name it as "Summary".
%\renewcommand{\abstractname}{Summary}

%%% Depth of section numbering
\setcounter{secnumdepth}{3}

%%% Depth of numbering in table of content
\setcounter{tocdepth}{2}

\usepackage[utf8]{inputenc}

\usepackage{indentfirst} % indent the first paragraph of a section

\usepackage{bookmark}
\usepackage[
  backend=biber,
  style=ieee,
  citestyle=numeric,
  giveninits=true,
  sorting=nyt,
  maxbibnames=99,
  dashed=false,
  doi=false]{biblatex}
\DeclareFieldFormat*{title}{``#1''\newunitpunct} % move comma outside the quotation mark
\addbibresource{abb.bib}
\addbibresource{references.bib}

\usepackage{microtype} % Better typography
\usepackage{tabulary} % Automatic table sizing
\usepackage{hhline}

\usepackage{enumitem}

\usepackage{hyperref}
\renewcommand{\sectionautorefname}{\S}
\let\subsectionautorefname\sectionautorefname
\let\subsubsectionautorefname\sectionautorefname
\renewcommand{\chapterautorefname}{Chapter}

\usepackage{listings}
\renewcommand{\lstlistingname}{Code}
\lstset{language=SQL, 
    morekeywords={ONLINE, WITHIN},
    frame=none,
    float,
    basicstyle=\ttfamily\normalsize,
    keywordstyle=\bfseries,
    numbers=none,
    showstringspaces=false,
    aboveskip=-2pt, %\smallskipamount
    belowskip=-4pt, %\smallskipamount
    xleftmargin=1em,
    escapeinside={(*@}{@*)},
    captionpos=b
}

\let\proof\relax
\let\endproof\relax
\usepackage{amsthm}
\theoremstyle{definition}
\newtheorem{defn}{Definition} \newcommand{\defnautorefname}{Definition}
\theoremstyle{plain}
\newtheorem{rul}{Rule} \newcommand{\rulautorefname}{Rule}
\newtheorem{thm}{Theorem} \newcommand{\thmautorefname}{Theorem}
\newtheorem{lemma}{Lemma} \newcommand{\lemmaautorefname}{Lemma}
\newtheorem{coroll}{Corollary} \newcommand{\corollautorefname}{Corollary}

\usepackage{mathtools}
\DeclarePairedDelimiter{\ceil}{\lceil}{\rceil}
\DeclarePairedDelimiter{\floor}{\lfloor}{\rfloor}
\DeclarePairedDelimiter{\vbar}{\vert}{\vert}
\DeclarePairedDelimiter{\vbarbar}{\Vert}{\Vert}
\DeclarePairedDelimiter{\parenLR}{\lparen}{\rparen}
\DeclarePairedDelimiter{\brackLR}{\lbrack}{\rbrack}
\DeclarePairedDelimiter{\braceLR}{\lbrace}{\rbrace}
\DeclarePairedDelimiter{\angleLR}{\langle}{\rangle}
\DeclareMathOperator*{\argmin}{arg\,min}
\DeclareMathOperator*{\argmax}{arg\,max}

\usepackage{pgf,interval}
\intervalconfig{soft open fences}
\newcommand{\intervalO}{\interval[open]}
\newcommand{\intervalOL}{\interval[open left]}
\newcommand{\intervalOR}{\interval[open right]}

% \usepackage[linesnumbered,ruled,vlined]{algorithm2e}
% \SetAlgorithmName{Algorithm}{Algorithm}{Algorithm}
% \newcommand{\AlgoFontSize}{\small} % \scriptsize \footnotesize \small \normalsize
% \IncMargin{0.5em}
% \SetCommentSty{textnormal}
% \SetNlSty{}{}{:}
% \SetAlgoNlRelativeSize{0}
% \SetKwInput{KwGlobal}{Global}
% \SetKwInput{KwPrecondition}{Precondition}
% \SetKwProg{Proc}{Procedure}{:}{}
% \SetKwProg{Func}{Function}{:}{}
% \SetKw{And}{and}
% \SetKw{Or}{or}
% \SetKw{To}{to}
% \SetKw{DownTo}{downto}
% \SetKw{Break}{break}
% \SetKw{Continue}{continue}
% \SetKw{SuchThat}{\textit{s.t.}}
% \SetKw{WithRespectTo}{\textit{wrt}}
% \SetKw{Iff}{\textit{iff.}}
% \SetKw{MaxOf}{\textit{max of}}
% \SetKw{MinOf}{\textit{min of}}
% \SetKwBlock{Match}{match}{}{}

\usepackage{array}
\newcolumntype{L}[1]{>{\raggedright\let\newline\\\arraybackslash\hspace{0pt}}m{#1}}
\newcolumntype{C}[1]{>{\centering\let\newline\\\arraybackslash\hspace{0pt}}m{#1}}
\newcolumntype{R}[1]{>{\raggedleft\let\newline\\\arraybackslash\hspace{0pt}}m{#1}}

\usepackage{color}
\usepackage{xcolor}
\usepackage{soul}
\soulregister\cite7
\soulregister\ref7
\soulregister\pageref7
\soulregister\autoref7
\soulregister\eqref7
\newcommand{\hlred}[2][Lavender]{{\sethlcolor{#1}\hl{#2}}}
\newcommand{\hlblue}[2][SkyBlue]{{\sethlcolor{#1}\hl{#2}}}
\newcommand{\hlyellow}[2][GreenYellow]{{\sethlcolor{#1}\hl{#2}}}
\newcommand{\hlgreen}[2][YellowGreen]{{\sethlcolor{#1}\hl{#2}}}

\let\originaleqref\eqref
\renewcommand{\eqref}{Equation~\ref}

%%% Use this command when you need to add a hyphen with hyphenation 
%%% enabled for the individual compound words
\newcommand{\zz}{-\nolinebreak\hspace{0pt}}

%%% Handy commands for image/figure insertion
\newcommand*{\RootPicDir}{pic}
\newcommand*{\PicDir}{\RootPicDir}
\newcommand*{\ResetPicDir}{\renewcommand*{\PicDir}{\RootPicDir}}
\newcommand*{\SetPicSubDir}[1]{\renewcommand*{\PicDir}{\RootPicDir /#1}}
\newcommand*{\Pic}[2]{\PicDir /#2.#1}

\newcommand*{\RootExpDir}{exp}
\newcommand*{\ExpDir}{\RootExpDir}
\newcommand*{\ResetExpDir}{\renewcommand*{\ExpDir}{\RootExpDir}}
\newcommand*{\SetExpSubDir}[1]{\renewcommand*{\ExpDir}{\RootExpDir /#1}}
\newcommand*{\Exp}[2]{\ExpDir /#2.#1}

\newcommand*{\BeforeCaptionVSpace}{1ex}
\newcommand*{\BeforeSubCaptionVSpace}{0.75ex}

%%% For example code used only
\usepackage{lipsum} % for generating dummy text
\newcommand{\CMD}[1]{\texttt{\string#1}} % for typing a command

% from AAAI paper
\usepackage{tikz}
\usetikzlibrary{positioning}
\usepackage{algorithm}
\usepackage{algpseudocode}
\usepackage{algorithmicx}

\newcommand{\tup}[1]{\langle #1 \rangle}
\renewcommand{\cal}[1]{\mathcal{#1}}
\newcommand{\R}{\mathbb{R}}
\newcommand{\eps}{\varepsilon}
\newcommand{\Ch}{\mathit{Ch}}
\newcommand{\ch}{\mathit{ch}}
\newcommand{\Av}{\mathit{Av}}
\newcommand{\av}{\mathit{av}}
\newcommand{\CC}{\mathit{CC}}
\newcommand{\SCC}{\mathit{SCC}}
\newcommand{\core}{\mathit{Core}}
\newcommand{\samples}{\omega}
\newcommand{\citenameyear}[1]{\citeauthor{#1}, (\citeyear{#1})}


\newtheorem{theorem}{Theorem}[section]
\newtheorem{proposition}[theorem]{Proposition}
\theoremstyle{definition}
\newtheorem{example}[theorem]{Example}
\newtheorem{definition}[theorem]{Definition}

